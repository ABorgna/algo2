\documentclass[10pt, a4paper]{article}
% especifico m�rgenes manualmente
\usepackage[paper=a4paper, left=1.5cm, right=1.5cm, bottom=1.5cm, top=3.5cm]{geometry}
% codificaci�n ISO-8859-1
\usepackage[latin1]{inputenc}
% separaci�n sil�bica en castellano
\usepackage[spanish]{babel}
% paquetes y caratula de algo2
\usepackage{aed2-symb,aed2-itef,aed2-tad,caratula}


\begin{document}

% Estos comandos deben ir antes del \maketitle
\materia{Algoritmos y Estructuras de Datos II} % obligatorio
\submateria{Primer Cuatrimestre de 2014} % opcional
\titulo{Trabajo Pr�ctico 1} % obligatorio
\subtitulo{Especificaci�n} % opcional
\grupo{Grupo 1} % opcional 

\integrante{Integrante 1}{Nro/YY}{mail@dc.uba.ar} % obligatorio 
\integrante{Integrante 2}{Nro/YY}{mail@dc.uba.ar} % obligatorio 
\integrante{Integrante 3}{Nro/YY}{mail@dc.uba.ar} % obligatorio 
\integrante{Integrante 4}{Nro/YY}{mail@dc.uba.ar} % obligatorio 

\maketitle

% compilar 2 veces para actualizar las referencias
\tableofcontents

\pagebreak
%\newpage


\section{Especificaci�n}
Esta es una especificaci�n del Trabajo Pr�ctico n�mero 1 del 1$^{er}$ cuatrimestre del 2014 presentada por la c�tedra para la realizaci�n del Trabajo Pr�ctico 2. Ver enunciado:\\
\verb+http://www.dc.uba.ar/materias/aed2/2014/1c/descargas/tps/tp1-enunciado/view+\\

El enunciado es ligeramente diferente, en donde se agrega lo siguiente:
\begin{itemize}
\item Las promesas se ejecutan �nicamente ante un cambio de cotizaci�n.
\item El orden para ejecutar las promesas pendientes ahora est� definido y es el siguiente:
 \begin{enumerate}
 \item Se ejecutan las de venta.
 \item En el momento de ejecutar promesas de compra se le da prioridad a los clientes que mayor cantidad total de acciones poseen.
 \end{enumerate}
\end{itemize}


%%%%%%%%%%%%%%%%%%%% Renombres
\section{Renombres de TADs}

\tadNombre{TAD Cliente} es \tadNombre{Nat}

\tadNombre{TAD Dinero} es \tadNombre{Nat}

\tadNombre{TAD Nombre} es \tadNombre{String}

\tadNombre{TAD TipoPromesa} es \tadNombre{Enum \{compra, venta\}} 

\tadNombre{TAD ParPC} es \tadNombre{Tupla(Promesa,Cliente)}



%%%%%%%%%%%%%%%%%%%% Promesa
\section{TAD \tadNombre{Promesa}}

\begin{tad}{\tadNombre{Promesa}}
\tadIgualdadObservacional{p}{p'}{promesa}{título(p) = título(p') $\land$ tipo(p) = tipo(p') $\land$ limite(p) = limite(p') $\land$ cantidad(p) = cantidad(p')}

\tadGeneros{promesa}
\tadExporta{promesa, generadores, observadores, otras operaciones}
\tadUsa{\tadNombre{TipoPromesa}, \tadNombre{Dinero}, \tadNombre{Nombre}, \tadNombre{Nat}}

%%%%%%%%%%%%%%%%%%%%%%%%%%%%%%%%%%%%%
\tadObservadores
\tadAlinearFunciones{cantidad~}{promesa,nat}

\tadOperacion{título}{promesa}{nombre}{}
\tadOperacion{tipo}{promesa}{tipoPromesa}{}
\tadOperacion{límite}{promesa}{dinero}{}
\tadOperacion{cantidad}{promesa}{nat}{}

%%%%%%%%%%%%%%%%%%%%%%%%%%%%%%%%%%%%%%%
\tadGeneradores
\tadAlinearFunciones{crearPromesa~}{nombre, tipoPromesa, dinero, nat}

\tadOperacion{crearPromesa}{nombre, tipoPromesa, dinero, nat}{promesa}{}

%%%%%%%%%%%%%%%%%%%%%%%%%%%%%%%%%%%%%%%
\tadOtrasOperaciones
\tadAlinearFunciones{promesasSobreTítulo}{nombre,tipoPromesa,conj(promesa)}

\tadOperacion{promesaEjecutable}{promesa,dinero,nat}{bool}{}

\tadOperacion{compraVenta}{nat,conj(promesa)}{nat}{}

\tadOperacion{promesasSobreTítulo}{nombre,tipoPromesa,conj(promesa)}{conj(promesa)}{}

%%%%%%%%%%%%%%%%%%%%%%%%%%%%%%%%%%%%%%%
\tadAxiomas[\paratodo{nombre}{t} \paratodo{nat}{n, m} \paratodo{tipoPromesa}{tipo}]
\tadAlinearAxiomas{cantidad(crearPromesa(t,tipo,n,mmm))}

% OB título
\tadAxioma{título(crearPromesa(t, tipo, n, m))}{t}

% OB tipo
\tadAxioma{tipo(crearPromesa(t, tipo, n, m))}{tipo}

% OB límite
\tadAxioma{límite(crearPromesa(t, tipo, n, m))}{n}

% OB cantidad
\tadAxioma{cantidad(crearPromesa(t, tipo, n, m))}{m}
\vskip12pt


% OO promesaEjecutable
\tadAlinearAxiomas{promesaEjecutable(p,cot,disp)}
\tadAxioma{promesaEjecutable(p,cot,disp)}
{
(tipo(p) = venta $\land$ cot < \ límite(p)) $\lor$ \\
\big(tipo(p) = compra $\land$ cot > \ límite(p) $\land$ disp $\geq$~cantidad(p)\big)
}
\vskip12pt


% OO compraVenta
\tadAlinearAxiomas{compraVenta(disp, ps)}
%
\tadAxioma{compraVenta(disp, ps)}{
\IF vacío?(ps) THEN disp ELSE
	{\IF tipo(dameUno(ps)) = compra
	THEN compraVenta(disp + cantidad(dameUno(ps)), sinUno(ps))
	ELSE compraVenta(disp - cantidad(dameUno(ps)), sinUno(ps))
	FI}
FI}
\vskip12pt

% OO promesasSobreTítulo
\tadAlinearAxiomas{promesasSobreTítulo(t1,tp,ps)}
\tadAxioma{promesasSobreTítulo(t1,tp,ps)}
{\IF vacío?(ps) THEN $\emptyset$
 ELSE promesasSobreTítulo(t1, tp, sinUno(ps)) $\cup$ \\
 	({\IF t1 = título(dameUno(ps)) $\land$ tp = tipo(dameUno(ps)) THEN \{dameUno(ps)\} ELSE $\emptyset$ FI})
 FI}

\end{tad}



%%%%%%%%%%%%%%% Tad Título
\section{TAD \tadNombre{Título}}

\begin{tad}{\tadNombre{Título}}
\tadIgualdadObservacional{t}{t'}{título}{nombre(t) = nombre(t') $\land$ cotización(t) = cotización(t') $\land$ \#máxAcciones(t) = \#máxAcciones(t') $\land$ \\ enAlza?(t) $\iff$ enAlza?(t') }

\tadGeneros{título}
\tadExporta{título, generadores, observadores, otras operaciones}
\tadUsa{\tadNombre{Nombre}, \tadNombre{Dinero}, \tadNombre{Nat}}

\tadObservadores
\tadAlinearFunciones{\#máxAcciones~}{título,nat}

\tadOperacion{nombre}{título}{nombre}{}
\tadOperacion{\#máxAcciones}{título}{nat}{}
\tadOperacion{cotización}{título}{dinero}{}
\tadOperacion{enAlza}{título}{bool}{}

%%%%%%%%%%%%%%%%%%%%%%%%%%%%%%%%%%%%%%%%%
\tadGeneradores
\tadAlinearFunciones{crearTítulo~}{nombre,dinero,nat}

\tadOperacion{crearTítulo}{nombre,dinero,nat}{título}{}
\tadOperacion{recotizar}{dinero,título}{título}{}

%%%%%%%%%%%%%%%%%%%%%%%%%%%%%%%%%%%%%%%%%
\tadOtrasOperaciones
\tadAlinearFunciones{cambiarCotización}{nombre/nomTit,dinero/cot,conj(título)/ts}


\tadOperacion{cotizaciónActual}{nombre/nomTit,conj(título)/ts}{nat}{(\existe{título}{t})(t $\in$ ts $\land$ nombre(t) = nomTit)}

\tadOperacion{cambiarCotización}{nombre/nomTit,dinero/cot,conj(título)/ts}{conj(título)}{(\existe{título}{t})(t $\in$ ts $\land$ nombre(t) = nomTit)}

\tadOperacion{límiteTenencia}{nombre/nomTit,conj(título)/ts}{nat}{(\existe{título}{t})(t $\in$ ts $\land$ nombre(t) = nomTit)}

\tadOperacion{títuloEnAlza}{nombre/nomTit,conj(título)/ts}{bool}{(\existe{título}{t})(t $\in$ ts $\land$ nombre(t) = nomTit)}


\medskip

%%%%%%%%%%%%%%%%%%%%%%%%%%%%%%%%%%%%%%%%%
\tadAxiomas[\paratodo{nombre}{s}  \paratodo{nat}{n,c,c'}]

\tadAlinearAxiomas{\#máxAcciones(crearTítulo(s,c,n))}

% OB nombre
\tadAxioma{nombre(crearTítulo(s,c,n))}{s}
\tadAxioma{nombre(recotizar(c,t))}{nombre(t)}

% OB \#máxAcciones
\tadAxioma{\#máxAcciones(crearTítulo(s,c,n))}{n}
\tadAxioma{\#máxAcciones(recotizar(c,t))}{\#máxAcciones(t)}

% OB cotización
\tadAxioma{cotización(crearTítulo(s,c,n))}{c}
\tadAxioma{cotización(recotizar(c,t))}{c}

% OB enAlza
\tadAxioma{enAlza(crearTítulo(s,c,n))}{true}
\tadAxioma{enAlza(recotizar(c,t))}
{
  c > \ cotización(t)
}

\vskip12pt


% OO cotizaciónActual
\tadAlinearAxiomas{cotizaciónActual(t1, ts)}
%
\tadAxioma{cotizaciónActual(t1, ts)}%
{\IF nombre(dameUno(ts)) = t1 THEN 
	cotización(dameUno(ts)) 
ELSE 
	cotizaciónActual(t1, sinUno(ts))
FI}
\vskip12pt


% OO cambiarCotización
\tadAlinearAxiomas{cambiarCotización(t1, cot, ts)}
%
\tadAxioma{cambiarCotización(t1, cot, ts)}%
{\IF vacío?(ts) THEN
	$\emptyset$
ELSE 
	{\IF nombre(dameUno(ts)) = t1 THEN 
		Ag(recotizar(cot,dameUno(ts))), sinUno(ts)) 
	ELSE 
		Ag(dameUno(ts),cambiarCotización(t1, cot, sinUno(ts)))
	FI}
FI}
\vskip12pt


% OO límiteTenencia
\tadAlinearAxiomas{límiteTenencia(t1,ts)}
%
\tadAxioma{límiteTenencia(t1,ts)}{
\IF nombre(dameUno(ts)) = t1 THEN \#máxAcciones(dameUno(ts)) ELSE límiteTenencia(t1,sinUno(ts)) FI}


% OO títuloEnAlza
\tadAlinearAxiomas{títuloEnAlza(t1,ts)}
%
\tadAxioma{títuloEnAlza(t1,ts)}{
\IF nombre(dameUno(ts)) = t1 THEN enAlza(dameUno(ts)) ELSE títuloEnAlza(t1,sinUno(ts)) FI}

\end{tad}



%%%%%%%%%%%%%%%%%%%% Wolfie
\section{TAD \tadNombre{Wolfie}}

\begin{tad}{\tadNombre{Wolfie}}
\tadGeneros{wolfie}
\tadExporta{wolfie, generadores, observadores, enAlza}
\tadUsa{\tadNombre{Bool}, \tadNombre{Nat}, \tadNombre{Dinero}, \tadNombre{Promesa}, \tadNombre{Cliente}, \tadNombre{Título}, \tadNombre{Conjunto($\alpha$)}, \tadNombre{Nombre}, \tadNombre{TipoPromesa}, \tadNombre{parPC}}

\tadIgualdadObservacional{w1}{w2}{wolfie}{clientes(w1) = clientes(w2) $\land$ títulos(w1) = títulos(w2) \yluego ($\forall$ c:cliente, t:título) ((c $\in$ clientes(w1) $\land$ t $\in$ títulos(w1)) $\impluego$ (promesasDe(c, w1) = promesasDe(c, w2)) $\land$ accionesPorCliente(c, nombre(t), w1) = accionesPorCliente(c, nombre(t), w2)) }

%%%%%%%%%%%%%%%%%%%%%%%%%%%%%%%%%%%%%%%

\tadObservadores
\tadAlinearFunciones{accionesPorCliente}{cliente/c,nombre/nomTit,wolfie/w}

\tadOperacion{clientes}{wolfie}{conj(cliente)}{}
\tadOperacion{títulos}{wolfie}{conj(título)}{}
\tadOperacion{promesasDe}{cliente/c,wolfie/w}{conj(promesa)}
  {c $\in$ clientes(w)}
\tadOperacion{accionesPorCliente}{cliente/c,nombre/nomTit,wolfie/w}{nat}
{c $\in$ clientes(w) $\land$ (\existe{título}{t})(t $\in$ títulos(w) $\land$ nombre(t) = nomTit)}

%%%%%%%%%%%%%%%%%%%%%%%%%%%%%%%%%%%%%%%
\tadGeneradores
\tadAlinearFunciones{actualizarCotización~}{título/t,nat/cot,wolfie/w,wolfie}

\tadOperacion{inaugurarWolfie}{conj(cliente)/cs}{wolfie}{$\neg$ vacío(cs)}
\tadOperacion{agregarTítulo}{título/t,wolfie/w}{wolfie}
{(\paratodo{título}{t_2})($t_2 \in$ títulos(w) $\implies$ nombre(t) $\neq$ nombre($t_2$))}

\tadOperacion{actualizarCotización}{nombre/nomTit,nat/cot,wolfie/w}{wolfie}{(\existe{título}{t})(t $\in$ títulos(w) $\land$ nombre(t) = nomTit)}

\tadOperacion{agregarPromesa}{cliente/c, promesa/p, wolfie/w}{wolfie}{(\existe{título}{t})(t $\in$ títulos(w) $\land$ nombre(t) = título(p)) $\land$ c $\in$ clientes(w) $\yluego$ ($\forall p_2$: promesa)($p_2$ $\in$ promesasDe(c, w) $\implies$ (título(p) $\neq$ título($p_2$) $\lor$ tipo(p) $\neq$ tipo($p_2$)) ) $\land$ (tipo(p) = vender $\implies$ accionesPorCliente(c, título(p), w) $\geq$ cantidad(p) ) )}

%%%%%%%%%%%%%%%%%%%%%%%%%%%%%%%%%%%%%%%
\tadOtrasOperaciones
\tadAlinearFunciones{promesasAEjecutarPorCliente}{nombre/nomTit,dinero/cot,cliente/c,wolfie/w}

\tadOperacion{enAlza}{nombre/nomTit,wolfie/w}{bool}{(\existe{título}{t})(t $\in$ títulos(w) $\land$ nombre(t) = nomTit)}


\tadOperacion{accDisponibles}{nombre/nomTit,wolfie/w}{nat}{(\existe{título}{t})(t $\in$ títulos(w) $\land$ nombre(t) = nomTit)}
\tadOperacion{tenenciasDeTodos}{nombre/nomTit,conj(cliente)/cs,wolfie/w}{nat}{cs $\subseteq$ clientes(w) $\land$ (\existe{título}{t})(t $\in$ títulos(w) $\land$ nombre(t) = nomTit)}
\tadOperacion{promesasAEjecutarPorCliente}{nombre/nomTit,dinero/cot,cliente/c,wolfie/w}{conj(promesa)}{c $\in$ clientes(w) $\land$ (\existe{título}{t})(t $\in$ títulos(w) $\land$ nombre(t) = nomTit)}
\tadOperacion{filtrarPorCliente}{cliente,conj(parPC)}{conj(promesa)}{}

\tadOperacion{ventasAEjecutar}{nombre/nomTit,dinero/cot,wolfie/w}{conj(parPC)}{(\existe{título}{t})(t $\in$ títulos(w) $\land$ nombre(t) = nomTit)}


\tadOperacion{comprasAEjecutar}{nombre/nomTit,dinero/cot,wolfie/w}{conj(parPC)}{(\existe{título}{t})(t $\in$ títulos(w) $\land$ nombre(t) = nomTit)}

\tadOperacion{promesasAEjecutar}{dinero/cot,nat/disp,secu(parPC)/pends}{conj(parPC)}{}

\tadOperacion{aparearPC}{nombre,tipoPromesa,conj(cliente),wolfie}{conj(parPC)}{}

\tadOperacion{accVendidas}{conj(parPC)}{nat}{}

\tadOperacion{promesasOrdenAccClie}{conj(parPC)/cs,conj(parPC)/vs,wolfie/w}{secu(parPC)}{}

\tadOperacion{ordenSecu}{conj(parPC),secu(cliente))}{secu(parPC)}{}

\tadOperacion{estaClie?}{cliente/c,conj(parPC)/ps}{bool}{$\neg$ vacío(ps)} 

\tadOperacion{damePromesa}{cliente/c,conj(parPC)/ps}{promesa}{estaClie?(c,ps)} 

\tadOperacion{ordenarPorAcciones}{dicc(cliente $\text{,}$ nat)}{secu(cliente)}{}

\tadOperacion{máxAcc}{dicc(cliente $\text{,}$ nat)/dc}{cliente}{$\neg$ vacío(dc)}

\tadOperacion{máxAccAux}{conj(cliente)/cs,dicc(cliente $\text{,}$ nat)/dc}{tupla(cliente,nat)}{$\neg$ vacío(dc) $\land_L$ cs $\subseteq$ claves(dc)}

\tadOperacion{totalAccClie}{conj(parPC)/vs,wolfie/w}{dicc(cliente,nat)}{}

\tadOperacion{totalAccClieAux}{conj(parPC)/cs,conj(parPC)/vs,wolfie/w}{dicc(cliente,nat)}{}

\tadOperacion{accTotalClie}{cliente/c,wolfie/w}{nat}{c $\in$ clientes(w)}

\tadOperacion{accTotalClieAux}{cliente/c,conj(título)/ts,wolfie/w}{nat}{c $\in$ clientes(w) $\land$ ts $\subseteq$ títulos(w)}

\tadOperacion{accVendidasClie}{conj(parPC)/ps,cliente/c}{nat}{}

\medskip
%\pagebreak
%%%%%%%%%%%%%%%%%%%% AXIOMAS Wolfie

\tadAxiomas[\paratodo{conj(cliente)}{cs} \paratodo{wolfie}{w} \paratodo{título}{t} \paratodo{nombre}{t1,t2} \paratodo{cliente}{c, cl} \paratodo{promesa}{p} \paratodo{nat}{cot} \paratodo{conj(título)}{ts} \paratodo{conj(promesa)}{ps} \paratodo{nat}{disp} \paratodo{tipoPromesa}{tp}]

% OB clientes
\tadAlinearAxiomas{clientes(actualizarCotización(t1,cot,w))(clientes(w))}
%
\tadAxioma{clientes(inaugurarWolfie(cs))}{cs}
\tadAxioma{clientes(agregarTítulo(t, w))}{clientes(w)}
\tadAxioma{clientes(actualizarCotización(t2, cot, w))}{clientes(w)}
\tadAxioma{clientes(agregarPromesa(c, p, w))}{clientes(w)}
\vskip12pt

% OB títulos
\tadAlinearAxiomas{títulos(actualizarCotización(t1,cot,w))(títulos(w))}
%
\tadAxioma{títulos(inaugurarWolfie(cs))}{$\emptyset$}
\tadAxioma{títulos(agregarTítulo(t, w))}{Ag(t, títulos(w))}
\tadAxioma{títulos(actualizarCotización(t1, cot, w))}{cambiarCotización(t1,c,títulos(w))}
\tadAxioma{títulos(agregarPromesa(c, p, w))}{títulos(w)}
\vskip12pt


% OB promesasDe
\tadAlinearAxiomas{promesasDe(c, actualizarCotización(t1, cot, w))}
%
\tadAxioma{promesasDe(c, inaugurarWolfie(cs))}{$\emptyset$}
%
\tadAxioma{promesasDe(c, agregarTítulo(t, w))}{promesasDe(c, w)}
%
\tadAxioma{promesasDe(c, agregarPromesa(cl, p, w))}
{
promesasDe(c,w) $\cup$ {\IF c = cl THEN\{p\} ELSE $\emptyset$ FI}
}
%
\tadAxioma{promesasDe(c, actualizarCotización(t1, cot, w))}{promesasDe(c, w) - \\
promesasAEjecutarPorCliente(t1, cot, c, w)}
\vskip12pt

% OO enAlza
\tadAlinearAxiomas{tenenciasDeTodos(t1,cs,w)}
\tadAxioma{enAlza(t1,w)}{
títuloEnAlza(t1,títulos(w))
}
\vskip12pt

% OO accDisponibles
\tadAlinearAxiomas{tenenciasDeTodos(t1,cs,w)}
\tadAxioma{accDisponibles(t1,w)}{
límiteTenencia(t1,títulos(w)) - tenenciasDeTodos(t1,clientes(w),w)
}
\vskip12pt


% OO tenenciasDeTodos
\tadAxioma{tenenciasDeTodos(t1,cs,w)}{
\IF vacío?(cs) THEN 0 ELSE accionesPorCliente(dameUno(cs),t1,w) + \\
tenenciasDeTodos(t1,sinUno(cs),w) FI
}
\vskip12pt


% OO promesasAEjecutarPorCliente
\tadAlinearAxiomas{promesasAEjecutarPorCliente(t1,cot,c,w)}
\tadAxioma{promesasAEjecutarPorCliente(t1,cot,c,w)}{
  filtrarPorCliente\big(c, (ventasAEjecutar(t1,cot,w) $\cup$ \\ \hspace*{8.5em}  comprasAEjecutar(t1,cot,w)) \big)
}
\vskip12pt


% OO filtrarPorCliente
\tadAlinearAxiomas{filtrarPorCliente(c, ps)}
\tadAxioma{filtrarPorCliente(c, ps)}{
\IF vacío?(ps) THEN $\emptyset$ ELSE
	filtrarPorCliente(c,sinUno(ps)) $\cup$  \\ {\big(\IF $\Pi_2$(dameUno(ps)) = c THEN \{$\Pi_1$(dameUno(ps))\} ELSE $\emptyset$ FI\big)
	} 
FI}
\vskip12pt

%% ventasAEjecutar
\tadAlinearAxiomas{ventasAEjecutar(t1,cot,w)}
\tadAxioma{ventasAEjecutar(t1,cot,w)}{
promesasAEjecutar(cot,0,conjASecu(aparearPC(t1,venta,clientes(w),w)))
}
\vskip12pt

%% comprasAEjecutar
\tadAlinearAxiomas{comprasAEjecutar(t1,cot,w)}
\tadAxioma{comprasAEjecutar(t1,cot,w)}{
promesasAEjecutar\big(cot, \\
\hspace*{1em} (accDisponibles(t1,w) + accVendidas(ventasAEjecutar(t1,cot,w))),\\
\hspace*{1em} 
 promesasOrdenAccClie\big(aparearPC(t1,compra,clientes(w),w),\\
 \hspace*{11.5em}  ventasAEjecutar(t1,cot,w),w\big)\big)
}
\vskip12pt


% OO aparearPC
\tadAlinearAxiomas{aparearPC(t1,tp,cs,w)}
\tadAxioma{aparearPC(t1,tp,cs,w)}
{
\IF vacío?(cs) THEN $\emptyset$
ELSE
aparearPC(t1,tp,sinUno(cs),w) $\cup$ \\
  \big({\IF vacío?(promesasSobreTítulo(t1, tp, promesasDe(dameUno(cs),w))) THEN $\emptyset$
  ELSE \{<promesasSobreTítulo(t1,tp,promesasDe(dameUno(cs),w)),dameUno(cs)>\}
  FI}\big)
FI}
\vskip12pt


% OO promesasAEjecutar
\tadAlinearAxiomas{promesasAEjecutar(cot,disp,pends)}
\tadAxioma{promesasAEjecutar(cot,disp,pends)}
{ 
{\IF vacía?(pends) THEN $\emptyset$ 
  ELSE {\IF promesaEjecutable($\Pi_1$(prim(pends)),cot,disp) THEN \{prim(pends)\} $\cup$\\
  			\big(
  			{\IF tipo($\Pi_1$(prim(pends))) = compra
  			  THEN
  				promesasAEjecutar(cot,\\ 
  				\hspace*{1em} disp - cantidad($\Pi_1$(prim(pends))),\\
  				\hspace*{1em} fin(pends))
			 ELSE
			 	promesasAEjecutar(cot,\\
			 	\hspace*{1em} disp + cantidad($\Pi_1$(dameUno(pends))),\\ 
			 	\hspace*{1em} fin(pends))
			 FI}
			 \big)
  		ELSE
  			promesasAEjecutar(cot, disp, fin(pends))
  		FI}
  FI}
}
\vskip12pt


% OO accVendidas
\tadAlinearAxiomas{accVendidas(ps)}
\tadAxioma{accVendidas(ps)}
{\IF vacío?(ps) THEN 0 ELSE
cantidad($\Pi_1$(dameUno(ps))) + accVendidas(sinUno(ps))
FI
}
\vskip12pt


% OO promesasOrdenAccClie
\tadAlinearAxiomas{promesasOrdenAccClie(cs,vs,w)}
\tadAxioma{promesasOrdenAccClie(cs,vs,w)}{
ordenSecu(cs, ordenarPorAcciones(totalAccClie(vs,w)))
}
\vskip12pt

% OO ordenSecu
\tadAlinearAxiomas{ordenSecu(ps,sc)}
\tadAxioma{ordenSecu(ps,sc)}{
\IF vacía?(sc) THEN <>
ELSE
 {\IF estaClie?(prim(sc),ps) 
  THEN
  	damePromesa(prim(sc),ps) $\bullet$ <>
  ELSE
  	<>
  FI
  \& ordenSecu(ps,fin(sc))
 }
FI
}
\vskip12pt


% OO estaClie?
\tadAlinearAxiomas{estaClie?(c,ps)}
\tadAxioma{estaClie?(c,ps)}{
$\Pi_2$(dameUno(ps)) = c $\lor_L$ estaClie?(c,sinUno(ps))
}
\vskip12pt


% OO damePromesa
\tadAlinearAxiomas{damePromesa(c,ps)}
\tadAxioma{damePromesa(c,ps)}{
\IF $\Pi_2$(dameUno(ps)) = c THEN 
 $\Pi_1$(dameUno(ps)) 
ELSE
  damePromesa(c,sinUno(ps))
FI
}
\vskip12pt

% OO ordenarPorAcciones
\tadAlinearAxiomas{ordenarPorAcciones(dc)}
\tadAxioma{ordenarPorAcciones(dc)}{
\IF vacío?(dc) THEN vacío
ELSE
máxAcc(dc) $\bullet$ ordenarPorAcciones(borrar(máxAcc(dc),dc))
FI
}
\vskip12pt

% OO máxAcc
\tadAlinearAxiomas{máxAccAux(cs,dc)}
\tadAxioma{máxAcc(dc)}{
$\Pi_1$(máxAccAux(claves(dc),dc))
}

% OO máxAccAux
\tadAxioma{máxAccAux(cs,dc)}{
\IF vacío(sinUno(cs)) THEN <dameUno(cs), obtener(dameUno(cs),dc)>
ELSE
{\IF obtener(dameUno(cs),dc) > \ $\Pi_2$(máxAccAux(sinUno(cs),dc))
THEN
<dameUno(cs), obtener(dameUno(cs),dc)>
ELSE
máxAccAux(sinUno(cs),dc)
FI
}
FI
}
\vskip12pt


% OO totalAccClie
\tadAlinearAxiomas{totalAccClieAux(cs,vs,w)}
\tadAxioma{totalAccClie(vs,w)}{
totalAccClieAux(clientes(w),vs,w)
}

% OO totalAccClieAux
\tadAxioma{totalAccClieAux(cs,vs,w)}{
\IF vacío?(cs) THEN vacío? 
ELSE 
definir\big(dameUno(cs),\\
\hspace*{1em} (accTotalClie(dameUno(cs),w)+accVendidasClie(vs,dameUno(cs))), 
\hspace*{1em} totalAccClieAux(sinUno(cs),vs,w)\big)
FI
}
\vskip12pt

% OO accTotalClie
\tadAlinearAxiomas{accTotalClieAux(c,ts,w)}
\tadAxioma{accTotalClie(c,w)}{
accTotalClieAux(c,títulos(w),w)
}

% OO accTotalClieAux
\tadAxioma{accTotalClieAux(c,ts,w)}{
\IF vacío?(ts) THEN 0
ELSE
accionesPorCliente(c,dameUno(ts),w)+accTotalClieAux(c,sinUno(ts),w)
FI
}
\vskip12pt


% OO accVendidasClie
\tadAlinearAxiomas{accVendidasClie(ps,c)}
\tadAxioma{accVendidasClie(ps,c)}
{\IF vacío?(ps) THEN 0 ELSE
accVendidasClie(c,sinUno(ps)) + \\
 {\IF c = $\Pi_2$(dameUno(ps)) THEN
 cantidad($\Pi_1$(dameUno(ps)))
 ELSE 0
 FI}
FI
}
\vskip12pt



% OB accionesPorCliente
\tadAlinearAxiomas{accionesPorCliente(c,t1,actualizarCotización(t2,cot,w))}
%
\tadAxioma{accionesPorCliente(c, t1, agregarTítulo(t2, cot, w))}
{\IF t1 = t2 THEN 0 ELSE accionesPorCliente(c, t1, w) FI}
%
\tadAxioma{accionesPorCliente(c, t1, agregarPromesa(cl, p, w))}
{accionesPorCliente(c,t1,w)}
\vskip12pt
%
%
\tadAxioma{accionesPorCliente(c,t1,actualizarCotización(t2,cot,w))}
{\IF t1 = t2 THEN
  compraVenta\big(accionesPorCliente(c,t1,w),\\
  \hspace*{1em} promesasAEjecutarPorCliente(t1,cot,c,w)\big)
  ELSE
  accionesPorCliente(c,t1,w)
 FI}


\end{tad}








\end{document}
