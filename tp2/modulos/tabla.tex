\section{Tabla}

\subsection{Interfaz}

\iusa{}
\iseexplica{Tabla}
\igenero{tabla}

\ioperaciones

\operacion{NuevaTabla}
{   \param{in}{$nombre$}{$string$},
    \param{in}{$claves$}{$conj(campos)$},
    \param{in}{$columnas$}{$registro$}} % parametros
{tabla} % tipo res
{$\neg\emptyset?(claves) \land claves \subseteq campos(columnas)$} % pre
{res \igobs nuevaTabla()} % post
{\bigo(1)} % complejidad
{} % aliasing
{} % descripcion

\operacion{AgregarRegistro}
{   \param{in/out}{$t$}{$tabla$},
    \param{in}{r}{$registro$}}
{}
{$t \igobs t_0 \land campos(r) \igobs campos(t) \land puedoInsertar?(r,t)$}
{$t$ \igobs agregarRegistro(r,$t_0$)}
{\bigo(in)}
{}
{in es O(log(n)) en promedio si hay un indice sobre un campo de tipo NAT, O(1) sino}

\operacion{BorrarRegistro}
{   \param{in/out}{$t$}{$tabla$},
    \param{in}{$crit$}{$registro$}}
{}
{$t \igobs t_0 \land \#campos(crit) = 1 \land_L dameUno(campos(crit)) \in claves(t)$}
{$t$ \igobs borrarRegistro($crit,t_0$)}
{\bigo(in)}
{}
{in es O(log(n)) en promedio si hay un indice sobre un campo de tipo NAT, O(n) sino}

\operacion{Indexar}
{   \param{in/out}{$t$}{$tabla$},
    \param{in}{c}{$campo$}}
{}
{$t \igobs t_0 \land puedeIndexar(c,t)$}
{$t$ \igobs indexar($c,t_0$)}
{\bigo(in)}
{}
{in es O(log(n)) en promedio si hay un indice sobre un campo de tipo NAT, O(n) sino}

\operacion{Nombre}
{   \param{in}{$t$}{$tabla$}}
{string}
{true}
{$res \igobs nombre(t)$}
{\bigo(1)}
{}
{}

\operacion{Claves}
{   \param{in}{$t$}{$tabla$}}
{conj(campo)}
{true}
{$res \igobs claves(t$}
{\bigo(1)}
{}
{}

\operacion{Indices}
{   \param{in}{$t$}{$tabla$}}
{conj(campo)}
{true}
{$res \igobs indices(t$}
{\bigo(1)}
{}
{}

\operacion{Campos}
{   \param{in}{$t$}{$tabla$}}
{conj(campo)}
{true}
{$res \igobs campos(t$}
{\bigo(1)}
{}
{}

\subsection{Representación}

-- todo: explicacion --

\serepresenta{tabla}{vec}
\donde{vec}{tupla($registros$: regs, $hayIndiceNat$: bool, $indiceNat$: campo, $indicesNat$: treeNat, \\
    $hayIndiceString$: bool, $indiceString$: campo, $indicesString$: treeString, \\
    $nombre$: nombreTabla, $campos$: registro, $claves$: conj($campo$), $accesos$: nat)}

\donde{regs}{lista($registro$)}
\donde{treeNat}{diccAvl($nat$, itLista($registro$))}
\donde{treeString}{diccAvl($string$, itLista($registro$))}

\donde{campo}{string}
\donde{nombreTabla}{string}
\donde{registro}{diccAvl($campo$, $dato$)}

\subsubsection{Invariante de representación}

\begin{Rep}{$estr$}{$e$}
    \repfunc{Hay claves y están incluidas en campos}
        {\neg\emptyset?(e.claves) \land e.claves \subseteq claves(e.campos)}

    \repfunc{Los registros tienen los mismos campos y del mismo tipo}
        {(\forall r : registro) (r \in e.registros \implies claves(r) = claves(e.campos) \land_L \\
        \hspace*{2em} (\forall c : campo) (c \in claves(r) \implies_L \; tipo?(obtener(c,r)) = tipo?(obtener(c,e.campos))))}

    \repfunc{Si no hay indice de tipo nat, indicesNat está vacio}
        {\neg e.hayIndiceNat \implies \emptyset?(e.indicesNat)}

    \repfunc{Si hay indice de tipo nat, indiceNat es un campo válido e indicesNat tiene los registros referenciados por el índice}
        {e.hayIndiceNat \implies indiceNat \in claves(e.campos) \land_L Nat?(obtener(indiceNat,e.campos)) \land_L \\
        \hspace*{2em} long(e.registros) = \#claves(e.indicesNat) \land \\
        \hspace*{2em} (\forall r : registro) (r \in e.registros \implies def?(obtener(indiceNat,r), e.indicesNat) \land_L \\
        \hspace*{12em} haySiguiente?(obtener(obtener(indiceNat,r), e.indicesNat)) \land_L \\
        \hspace*{12em} Siguiente(obtener(obtener(indiceNat,r), e.indicesNat)) \igobs r)}

    \repfunc{Si no hay indice de tipo string, indicesString está vacio}
        {\neg e.hayIndiceString \implies \emptyset?(e.indicesString)}

    \repfunc{Si hay indice de tipo string, indiceString es un campo válido e indicesString tiene los registros referenciados por el índice}
        {e.hayIndiceString \implies indiceString \in claves(e.campos) \land_L String?(obtener(indiceString,e.campos)) \land_L \\
        \hspace*{2em} long(e.registros) = \#claves(e.indicesString) \land \\
        \hspace*{2em} (\forall r : registro) (r \in e.registros \implies def?(obtener(indiceString,r), e.indicesString) \land_L \\
        \hspace*{12em} haySiguiente?(obtener(obtener(indiceString,r), e.indicesString)) \land_L \\
        \hspace*{12em} Siguiente(obtener(obtener(indiceString,r), e.indicesString)) \igobs r)}

    \repfunc{Hay por lo menos tantos accesos como registros insertados}
        {e.accesos \ge \#e.registros}

\end{Rep}

\subsubsection{Función de abstracción}

\begin{ABS}{$e$}{$estr$}{$t$}{$tabla$}
    \absfunc{}{e.nombre \igobs nombre(t) \; \land}
    \absfunc{}{e.claves \igobs claves(t) \; \land}
    \absfunc{}{e.accesos \igobs cantidadDeAccesos(t) \; \land}
    \absfunc{}
        {(\IFM e.hayIndiceNat THEN Ag(indiceNat, \emptyset) ELSE \emptyset FI) \: \cup \\
        (\IFM e.hayIndiceString THEN Ag(indiceString, \emptyset) ELSE \emptyset FI) \igobs indices(t) \; \land}
    \absfunc{}{claves(e.campos) \igobs campos(t) \; \land_L}
    \absfunc{}{(\forall c : campo) \: (c \in claves(e.campos) \implies_L \: tipo?(obtener(c,e.campos)) \igobs tipoCampo(c,t)) \; \land}
    \absfunc{}{(\forall r : registro) \: esta?(r, e.registros) \iff r \in registros(t)}
\end{ABS}

\subsection{Algoritmos}

\subsection{Servicios usados}

