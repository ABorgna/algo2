\section{diccTrie($\alpha$)}

El módulo diccTrie provee un diccionario con claves de tipo String y acceso, inserción y borrado en \bigo($L$), donde $L$ es el largo máximo de las claves.

\subsection{Interfaz}

\begin{iparamformales}{$\alpha$}

\end{iparamformales}

\iusa{}
\iseexplica{Diccionario($string, \alpha$)}
\igenero{diccTrie($\alpha$)}
\ioperaciones

\operacion{NuevoDiccTrie}
{}
{diccTrie($\alpha$)}
{true}
{$res \igobs vacio$}
{\bigo(1)}
{}
{Crea un diccionario vacio}

\operacion{Definir}
{   \param{in/out}{$d$}{diccTrie($\alpha$)},
    \param{in}{$c$}{string},
    \param{in}{$v$}{$\alpha$}}
{}
{$d \igobs d_0 \land \neq def?(c, d)$}
{$d \igobs definir(c, v, d_0)$}
{\bigo($L$)}
{Modifica el diccionario agregando o reemplazando el significado de una clave 
    con un nuevo valor}

\operacion{Definido?}
{   \param{in}{$d$}{diccTrie($\alpha$)},
    \param{in}{$c$}{string}}
{bool}
{true}
{$res \igobs def?(c, d)$}
{\bigo($L$)}
{}
{Devuelve true si una clave se encuentra definida en el diccionario}

\operacion{Obtener}
{   \param{in}{$d$}{diccTrie($\alpha$)},
    \param{in}{$c$}{string}}
{$\alpha$}
{$def?(c, d)$}
{$res$ \igobs $obtener(c, d)$}
{\bigo($L$)}
{}
{Devuelve el significado definido para la clave $c$}

\operacion{Borrar}
{   \param{in/out}{$d$}{diccTrie($\alpha$)},
    \param{in}{$c$}{string}}
{}
{$d \igobs d_0 \land def?(c, d)$}
{$d \igobs borrar(c, d_0)$}
{\bigo($L$)}
{}
{Borra el significado asociado a la clave $c$}

\operacion{Maximo}
{   \param{in}{$d$}{diccLog($\alpha$)}}
{tupla(clave: string, significado: $\alpha$)}
{$\neg (d \igobs vacio)$}
{$siguiente(res) \igobs tupla(maximo(d), obtener(maximo(d)))$}
{\bigo(1)}
{}
{Obtiene una tupla de la clave y el significado del elemento con la clave 
    mas grande en el diccionario}

\operacion{Minimo}
{   \param{in}{$d$}{diccLog($\alpha$)}}
{tupla(clave: string, significado: $\alpha$)}
{$\neg (d \igobs vacio)$}
{$siguiente(res) \igobs tupla(minimo(d), obtener(minimo(d)))$}
{\bigo(1)}
{}
{Obtiene una tupla de la clave y el significado del elemento con la clave 
    mas pequeña en el diccionario}


\subsection{Representación}
\serepresenta{diccTrie($\alpha$)}{trie}
\donde{trie}{tupla(\\
    $raiz$: puntero(nodo),\\
    $minimo$: clavevalor,
    $maximo$: clavevalor }
\donde{nodo}{tupla(\\
    $valor$: $\alpha$,\\
    $hijos$: arreglo\_estático[256] (puntero(nodo)) )}
\donde{clavevalor}{tupla(\\
    $clave$: string,\\
    $valor$: $\alpha$ }
    


\subsubsection{Invariante de representación}
\begin{Rep}{$trie$}{$e$}
 	\repfunc{ Raiz no es NULL $\land_L$}
    {
        e.raiz!=null 
    }
    \repfunc{ Hay un único camino entre cada nodo y la raiz (no hay ciclos) $\land$ }
    {
    	noHayCiclos(e)
    }
   
    \repfunc{ Todos los nodos hojas (los que tienen todas las posiciones del arreglo hijos en NULL) deben tener un valor distinto de NULL $\land$}
    {
    	todasLasHojasTienenValor(e)     
    }
    \repfunc{ Si el nodo raiz tiene hijos entonces hay claves $\land$}
    {
    	tienehijos(e)\Rightarrow |e.claves|>0     
    }
    \repfunc{ una palabra se encuentra en la lista de claves si y solo si al recorrer el arbol con esas letras se llega a una posicion con valor distinto de null}
    {
        (\forall c: string) (c\in caminoANodos(e)) \Leftrightarrow  ((\exists i:nat)(i\in \lbrace[0..e.claves\rbrace)\Rightarrow (e.claves[i]=c))
    }
   

\end{Rep}

\subsubsection{Función de abstracción}


\subsection{Algoritmos}
\algoritmo{iNuevoDiccTrie}{
        }{diccTrie($\alpha$)}{\bigo(1)}{
    \var $n : nodo$    
    \State $ n \gets crearNodo()$                   		  			\comment \bigo(1)    
    \State $res.raiz \gets *n$                     						\comment \bigo(1)    
    \State $res.minimo.valor \gets NULL$                     			\comment \bigo(1)  
    \State $res.minimo.clave \gets NULL$                     			\comment \bigo(1) 
    \State $res.maximo.valor \gets NULL$                     			\comment \bigo(1)  
    \State $res.maximo.clave \gets NULL$                     			\comment \bigo(1)                             
}{6 * \bigo(1) = \bigo(1)}

\algoritmo{iDefinir}
{   \param{in/out}{$e$}{diccTrie($\alpha$)},
    \param{in}{$c$}{string},
    \param{in}{$v$}{$\alpha$}}
{}{}
{
	\var $i : nat$
	\State $ i \gets 0$  												\comment \bigo(1) 
	\var $p : puntero(nodo)$ 
	\State $ p \gets e.raiz$											\comment \bigo(1) 
	\var $n : nodo$ 
	\While{$i<(longitud(c))$}
   		\Statex             \comment El loop se repite longitud de la clave veces
     	 \If{$p.hijos[ord(s[i]] == NULL$}                     			\comment \bigo(1)
            \State $n \gets crearNodo()$      							\comment \bigo(1)
           	\State $p.hijos[ord(s[i]) \gets \&n$              			\comment \bigo(1)
     	 \EndIf
     	\State $p \gets p.hijos[ord(s[i]))$								\comment \bigo(1)
     	\State $i++$   													\comment \bigo(1) 
    \EndWhile		
	\State $*p.valor \gets v)$										\comment \bigo(1)
    \If{$c<trie.minimo.clave\vert\vert trie.maximo.clave==NULL $}                     					\comment \bigo(1)
           \State $trie.minimo.clave \gets c$     				 	\comment \bigo(1)
             \State $trie.minimo.valor \gets v$     				 \comment \bigo(1)
       \EndIf	
     \If{$c>trie.maximo.clave \vert\vert trie.maximo.clave==NULL $}                     					\comment \bigo(1)
           \State $trie.maximo.clave \gets c$     				 	\comment \bigo(1)
           \State $trie.maximo.valor \gets v$     				 	\comment \bigo(1)
      \EndIf														 

}	{  2* \bigo(1) + L * (5 * \bigo(1)) + 7 * \bigo(1) = \bigo(L) }

\algoritmo{iDefinido?}
{   \param{in/out}{$e$}{diccTrie($\alpha$)},
    \param{in}{$c$}{string}}
    {bool}
    {\bigo(L)}
{
	\var $i : nat$
	\State $ i \gets 0$  												\comment \bigo(1) 
	\var $p : puntero(nodo)$ 
	\State $ p \gets e.raiz$											\comment \bigo(1)  
	\State $ res \gets NULL$											\comment \bigo(1)  
	\While{$i<(longitud(c)) \&\& res==NULL$}
	\Statex             \comment El loop se repite longitud de la clave veces o menos en caso de que res se vuelva false
		 \If{$p.hijos[ord(s[i]] != NULL$}                     			\comment \bigo(1)
		 	\State $p \gets p.hijos[ord(s[i]))$							\comment \bigo(1)
		 \Else
            \State $res \gets false$  									\comment \bigo(1)
     	 \EndIf	
     	\State $i++$   													\comment \bigo(1) 
    \EndWhile		
	\If{$res == NULL$}                     							\comment \bigo(1)
		 \State $res \gets true$									\comment \bigo(1)	
	\EndIf												 

}	{  3* \bigo(1) + L * (4 * \bigo(1)) + 2 * \bigo(1) = \bigo(L) }


\algoritmo{iObtener}
{   \param{in/out}{$e$}{diccTrie($\alpha$)},
    \param{in}{$c$}{string}}
    {$\alpha$}
    {\bigo(L)}
{
	\var $i : nat$
	\State $ i \gets 0$  												\comment \bigo(1) 
	\var $p : puntero(nodo)$ 
	\State $ p \gets e.raiz$											\comment \bigo(1)  
	\While{$i<(longitud(c))$}
     	\Statex             \comment El loop se repite longitud de la clave veces
     	\State $p \gets p.hijos[ord(s[i]))$								\comment \bigo(1)
     	\State $i++$   													\comment \bigo(1) 
    \EndWhile		
	\State $res \gets (*p).valor)$										\comment \bigo(1)												 

}	{  2* \bigo(1) + L * (2 * \bigo(1)) + \bigo(1) = \bigo(L) }

\algoritmo{iBorrar}
{   \param{in/out}{$e$}{diccTrie($\alpha$)},
    \param{in}{$c$}{string}}
    {}
    {\bigo(L)}
{
	\var $i : nat$
	\State $ i \gets 0$  												\comment \bigo(1) 
	\var $p : puntero(nodo)$ 
	\State $ p \gets e.raiz$											\comment \bigo(1)  
	\var $pi: pila(tupla(punte: puntero(nodo), siguiente: nat))$ 
	\While{$i<(longitud(c)-1)$}
     	\Statex             \comment El loop se repite longitud de la clave veces
     	\State $p \gets p.hijos[ord(s[i]))$								\comment \bigo(1)
     	\State $apilar(pi, (p,s[i+1])) $								\comment \bigo(copy(tupla(punte: puntero(nodo), siguiente: nat))) 
     	\State $i++$   													\comment \bigo(1) 
    \EndWhile	
    \State $p \gets p.hijos[ord(s[i]))$								\comment \bigo(1)
     \State $apilar(pi, (p,NULL)) $							\comment \bigo(copy(tupla(punte: puntero(nodo), siguiente: nat)))	
   \While{$(\neq TieneHijos(*(tope(pi).punte))) \& \& ((*(tope(pi).punte)).clave==NULL)$}
     	\Statex             \comment El loop se repite longitud de la clave veces como mucho
 
     	\State $p \gets desapilar(pi)$								\comment \bigo(1)
     	  \State $(*(tope(pi).punte)).hijos[*(tope(pi).siguiente)] == NULL$							\comment \bigo(1)	
    \EndWhile	
    \If{$\neq tieneHijos(e.raiz)$}                     			\comment \bigo(1)
		 	\State $e.maximo \gets NULL$							\comment \bigo(1)
	\State $e.minimo \gets NULL$							\comment \bigo(1)
	\Else
         \State $ p \gets e.raiz$	\comment \bigo(1)
         \var $s : string$ 
         \While{$(*p).valor==NULL $}
     		\Statex             \comment El loop se repite L veces como mucho
     		\State $s \gets s+chr(MenorHijo(*p))$								\comment \bigo(1)	
     		\State $p \gets MenorHijo(*p)$								\comment \bigo(1)	
     				
    	\EndWhile		
    	\State $e.minimo.clave \gets s$								\comment \bigo(1)	
    	\State $e.minimo.valor \gets (*p).valor$								\comment \bigo(1)	
    	\While{$iTieneHijos(*p)$}
     		\Statex             \comment El loop se repite L veces como mucho
     		\State $s \gets s+chr(MayorHijo(*p))$								\comment \bigo(1)	
     		\State $p \gets MayorHijo(*p)$								\comment \bigo(1)	
     				
    	\EndWhile		
    	\State $e.maximo.clave \gets s$								\comment \bigo(1)	
    	\State $e.maximo.valor \gets (*p).valor$								\comment \bigo(1)	
    	
     \EndIf	 	
												 

}	{  2* \bigo(1) + L * (2 * \bigo(1) + \bigo(copy(tupla(punte: puntero(nodo), siguiente: nat)))) + \bigo(1) +\bigo(copy(tupla(punte: puntero(nodo), siguiente: nat))))+ L * 2 * \bigo(1)+ 5 * \bigo(1)== \bigo(L) en caso de que el arbol este cavio, el el otro caso es 
2* \bigo(1) + L * (2 * \bigo(1) + \bigo(copy(tupla(punte: puntero(nodo), siguiente: nat)))) + \bigo(1) +\bigo(copy(tupla(punte: puntero(nodo), siguiente: nat))))+ L * 2 * \bigo(1)+\bigo(1)+2* L * 2 * \bigo(1) +2 * 2 *big O(1) == \bigo(L)}

\algoritmo{iMaximo}{
\param{in/out}{$e$}{diccTrie($\alpha$)
        }{tupla(clave: string, significado: $\alpha$)}{\bigo(1)}{  
    \State $ res \gets e.maximo$                   		  			\comment \bigo(1)                             
}{ \bigo(1) = \bigo(1)}

\algoritmo{iMinimo}{
\param{in/out}{$e$}{diccTrie($\alpha$)
        }{tupla(clave: string, significado: $\alpha$)}{\bigo(1)}{  
    \State $ res \gets e.minimo$                   		  			\comment \bigo(1)                             
}{ \bigo(1) = \bigo(1)}





\subsubsection{Funciones auxiliares de los algoritmos}

\algoritmo{iCrearNodo}{
}{nodo}{\bigo(1)}
  {
    \var $d : arreglo_est\acute{a}tico[256] (puntero(nodo))$
    \var $i : nat$
    \State $i \gets 0)$                									\comment \bigo(1)

    \While{$i<256$}
     	\Statex             \comment El loop se repite 256 veces
     	\State $d[i] \gets NULL)$										\comment \bigo(1)
     	\State $i++$   													\comment \bigo(1) 
    \EndWhile						
    \State $res.hijos \gets d)$											\comment \bigo(1) 
    \State $res.valor \gets NULL)$   									\comment \bigo(1) 
}{   \bigo(1) + 256 * (2 * \bigo(1)) + 2 * \bigo(1) = \bigo(1) }

\algoritmo{iTieneHijos}{
\param{in}{$n$}{nodo}
}{bool}{\bigo(1)}
  {
    \var $i : nat$
    \State $i \gets 0)$                									\comment \bigo(1)
    \State $res \gets false$										\comment \bigo(1)
    \While{$i<256$}
     	\Statex             \comment El loop se repite 256 veces
 		\If{$n.hijos[i] != NULL$}                     			\comment \bigo(1)
		 	\State $res \gets true$										\comment \bigo(1)
     	 \EndIf	 													 
    \EndWhile						
}{  2* \bigo(1) + 256 * (2 * \bigo(1))  = \bigo(1) }

\algoritmo{iMenorHijo}{
\param{in}{$n$}{nodo}
}{bool}{\bigo(1)}
  {
    \var $i : nat$
    \State $i \gets 0)$                									\comment \bigo(1)
    \State $res \gets 256$												\comment \bigo(1)
    \While{$i<256$}
     	\Statex             \comment El loop se repite 256 veces
 		\If{$n.hijos[i] != NULL \&\& i<res$}                     			\comment \bigo(1)
		 	\State $res \gets i$										\comment \bigo(1)
     	 \EndIf	 													 
    \EndWhile						
}{  2* \bigo(1) + 256 * (2 * \bigo(1))  = \bigo(1) }

\algoritmo{iMayorHijo}{
\param{in}{$n$}{nodo}
}{bool}{\bigo(1)}
  {
    \var $i : nat$
    \State $i \gets 0)$                									\comment \bigo(1)
    \State $res \gets 0$												\comment \bigo(1)
    \While{$i<256$}
     	\Statex             \comment El loop se repite 256 veces
 		\If{$n.hijos[i] != NULL \&\& i>res$}                     			\comment \bigo(1)
		 	\State $res \gets i$										\comment \bigo(1)
     	 \EndIf	 													
    \EndWhile						
}{  2* \bigo(1) + 256 * (2 * \bigo(1))  = \bigo(1) }


\subsection{Servicios usados}


