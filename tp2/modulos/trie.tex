\section{diccTrie($\alpha$)}

El módulo diccTrie provee un diccionario con claves de tipo String y acceso, inserción y borrado en \bigo($L$), donde $L$ es el largo máximo de las claves.

\subsection{Interfaz}

\begin{iparamformales}{$\alpha$}

No es necesario pedir operaciónes de comparación sobre los significados.

\end{iparamformales}

\iusa{}
\iseexplica{Diccionario($string, \alpha$)}
\igenero{diccTrie($\alpha$)}
\ioperaciones

\operacion{NuevoDiccTrie}
{}
{diccTrie($\alpha$)}
{true}
{$res \igobs vacio$}
{\bigo(1)}
{}
{Crea un diccionario vacio}

\operacion{Definir}
{   \param{in/out}{$d$}{diccTrie($\alpha$)},
    \param{in}{$c$}{string},
    \param{in}{$v$}{$\alpha$}}
{}
{$d \igobs d_0 \land \neg def?(c,d)$}
{$d \igobs definir(c, v, d_0)$}
{\bigo($L$)}
{}
{Modifica el diccionario agregando o reemplazando el significado de una clave
    con un nuevo valor}

\operacion{Definido?}
{   \param{in}{$d$}{diccTrie($\alpha$)},
    \param{in}{$c$}{string}}
{bool}
{true}
{$res \igobs def?(c, d)$}
{\bigo($L$)}
{}
{Devuelve true si una clave se encuentra definida en el diccionario}

\operacion{Obtener}
{   \param{in}{$d$}{diccTrie($\alpha$)},
    \param{in}{$c$}{string}}
{$\alpha$}
{$def?(c, d)$}
{$res$ \igobs $obtener(c, d)$}
{\bigo($L$)}
{}
{Devuelve el significado definido para la clave $c$}

\operacion{Borrar}
{   \param{in/out}{$d$}{diccTrie($\alpha$)},
    \param{in}{$c$}{string}}
{}
{$d \igobs d_0 \land def?(c, d)$}
{$d \igobs borrar(c, d_0)$}
{\bigo($L$)}
{}
{Borra el significado asociado a la clave $c$}

\operacion{Maximo}
{   \param{in}{$d$}{diccLog($\alpha$)}}
{tupla(clave: string, significado: $\alpha$)}
{$\neg (d \igobs vacio)$}
{$res \igobs tupla(maximo(d), obtener(maximo(d)))$}
{\bigo(1)}
{}
{Obtiene una tupla de la clave y el significado del elemento con la clave
    mas grande en el diccionario}

\operacion{Minimo}
{   \param{in}{$d$}{diccLog($\alpha$)}}
{tupla(clave: string, significado: $\alpha$)}
{$\neg (d \igobs vacio)$}
{$res \igobs tupla(minimo(d), obtener(minimo(d)))$}
{\bigo(1)}
{}
{Obtiene una tupla de la clave y el significado del elemento con la clave
    mas pequeña en el diccionario}


\subsection{Representación}
\serepresenta{diccTrie($\alpha$)}{trie}
\donde{trie}{tupla(\\
    $raiz$: puntero(nodo),\\
    $minimo$: clavevalor,
    $maximo$: clavevalor }
\donde{nodo}{tupla(\\
    $valor$: $\alpha$,\\
    $hijos$: arreglo\_estático[256] (puntero(nodo)) )}
\donde{clavevalor}{tupla(\\
    $clave$: string,\\
    $valor$: $\alpha$ )}



\subsubsection{Invariante de representación}
\begin{Rep}{estrDic}{e}
    \repfunc{Existe un \'unico camino entre cada nodo y el nodo ra\'iz (no hay ciclos)}
        {\neg(a.raiz = NULL) \implies noHayCiclos( ag(e.raiz, \emptyset) , e.raiz)}
    \repfunc{Todos los nodos hojas, es decir, todos los que tienen su arreglo hijos con todas sus posiciones en NULL, tienen que tener un valor distinto de NULL.}
        {\neg(a.raiz = NULL) \implies todasLasHojasTienenValor(e)}
    \repfunc{Los nodos minimo y maximo son los correspondientes}
        {\neg(a.raiz = NULL) \implies (e.minimo==minimo(e.raiz) \land e.maximo==maximo(e.raiz))}

\end{Rep}

\subsubsection{Operaciones auxiliares del invariante de Representaci\'on}

  \tadOperacion{noHayCiclos}{conjunto(puntero(nodo)) /p, puntero(nodo) /q}{bool}{}
  \tadAxioma{noHayCiclos($p, q$)}{auxNoHayCiclos($p,q,0$)}
  \tadOperacion{auxNoHayCiclos}{conjunto(puntero(nodo)) /p, puntero(nodo) /q, nat /n}{bool}{}
  \tadAxioma{auxNoHayCiclos($p ,q , n$)}{\IF $(*q).hijos[n] \in p$ THEN false ELSE {\IF (*q).hijos[n]=NULL THEN {\IF $n<255$ $\land$ auxNoHayCiclos($p ,q , n+1$) THEN true ELSE {\IF n=255 THEN true ELSE false FI} FI} ELSE {\IF $n<255$ $\land$ noHayCiclos($ag(p, (*q).hijos[n]), (*q).hijos[n]$) $\land$ auxNoHayCiclos($p ,q , n+1$) THEN true ELSE {\IF n=255 $\land$ noHayCiclos($ag(p, (*q).hijos[n]), (*q).hijos[n]$) THEN true ELSE false  FI} FI} FI} FI}



  \tadOperacion{maximo}{puntero(nodo) /p}{clavevalor}{noHayCiclos( ag($\emptyset$ , p ) , p)}
  \tadAxioma{maximo($p$)}{\IF $\neg$tieneHijos(p) THEN (NULL,p.valor) ELSE (chr(auxMaximo(p,255))+maximo((*p).hijos[auxMaximo(p,255)]).clave), (*p).hijos[auxMaximo(p,255)]).valor) FI}
  \tadOperacion{auxMaximo}{puntero(nodo) /p, nat /n}{nat}{tieneHijos(p)}
  \tadAxioma{auxMaximo($p , n$)}{\IF (*p).hijos[n] = NULL THEN auxMinimo(p,n-1) ELSE n FI}

  \tadOperacion{minimo}{puntero(nodo) /p}{clavevalor}{noHayCiclos( ag($\emptyset$ , p ) , p)}
  \tadAxioma{minimo($p$)}{\IF $\neg$tieneHijos(p)  THEN (NULL,p.valor) ELSE {\IF (*p).valor = NULL THEN (chr(auxMinimo(p,0))+minimo((*p).hijos[auxMinimo(p,0)]).clave), (*p).hijos[auxMinimo(p,0)]).valor) ELSE (NULL,p.valor) FI} FI}
  \tadOperacion{auxMinimo}{puntero(nodo) /p, nat /n}{nat}{tieneHijos(p)}
  \tadAxioma{auxMinimo($p , n$)}{\IF (*p).hijos[n] = NULL THEN auxMinimo(p,n+1) ELSE n FI}

  \tadOperacion{todasLasHojasTienenValor}{trie /e}{bool}{noHayCiclos( ag($\emptyset$ , e.raiz ) , e.raiz)}
  \tadAxioma{todasLasHojasTienenValor($e$)}{{\IF e.raiz = NULL THEN true ELSE auxTodasLasHojasTienenValor($e.raiz, 0$) FI}}
  \tadOperacion{auxTodasLasHojasTienenValor}{puntero(nodo) /p, nat /n}{bool}{noHayCiclos( ag($\emptyset$ , p ) , p)}
  \tadAxioma{auxTodasLasHojasTienenValor($p, n$)}{
        \IF $\neg tieneHijos(p)$ THEN
            $\neg((*p).valor = NULL)$
        ELSE
            {\IF (*p).hijos[n]=NULL THEN
                ($n<255$ $\land$ auxTodasLasHojasTienenValor($p, n+1$)) $\lor$ n=255
            ELSE
                {\IF $n<255$ $\land_L$ auxTodasLasHojasTienenValor($(*p).hijos[n], 0$) $\land$ \\
                    \hspace*{6em} auxTodasLasHojasTienenValor($p, n+1$) THEN
                    true
                ELSE
                    (n=255 $\land_L$ auxTodasLasHojasTienenValor($(*p).hijos[n],0$))
                FI}
            FI}
        FI}


  \tadOperacion{tieneHijos}{puntero(nodo) /p}{bool}{}
  \tadAxioma{tieneHijos($p$)}{auxTieneHijos($p,0$)}
  \tadOperacion{auxTieneHijos}{puntero(nodo) /p, nat /n}{bool}{}
  \tadAxioma{auxTieneHijos($p, n$)}
        {\IF $n<255$ $\land$ (*p).hijos[n]=NULL $\land$ $\neg$auxTieneHijos($p, n+1$) THEN
            false
        ELSE
            $\neg$(n = 255 $\land$ (*p).hijos[n] = NULL)
        FI}






\subsubsection{Función de abstracción}

\begin{ABS}{e}{estrDicc}{d}{dicc(string,$\alpha$)}

     \absfunc{}
    {
        (\forall s : string) \; def?(s, d) \Leftrightarrow (\neg(e.raiz = NULL) \land_L existe(s, 0, e.raiz))\ \land_L
    }

    \absfunc{}
    {
        (\forall s : string) \; def?(s, d) \implies obtener(s, d) = obtener(s, 0, e.raiz)
    }
\end{ABS}

\tadOperacion{existe}
{string /s, nat /k,  puntero(nodo)/n}
{bool}
{}
\tadAxioma{existe(s, k, n)}{\IF (*n).hijos[ord(s[k])] = NULL THEN false ELSE {\IF k = long(s)-1 $\land$ $\neg$((*n).hijos[ord(s[k])] = NULL) THEN true ELSE existe(s,k+1, (*n).hijos[ord(s[k])] ) FI} FI}




\tadOperacion{obtener}
{string /s, nat /k,  puntero(nodo)/n}
{$\alpha$}
{existe(s, 0, n)}
\tadAxioma{obtener(s, 0, n)}{\IF k = long(s)-1  THEN (*((*n).hijos[ord(s[k])])).valor ELSE obtener(s,k+1, (*n).hijos[ord(s[k])]) FI}



\subsubsection{Representacion del iterador de Claves del diccTrie($\alpha$)}

\serepresenta{itClaves($\alpha$)}{puntero(nodo)}
Su Rep y Abs son los de itLista($\alpha$) definido en el apunte de iteradores para el modulo Lista Enlazada.

\subsection{Algoritmos}
\algoritmo{iNuevoDiccTrie}{
        }{diccTrie($\alpha$)}{\bigo(1)}{
    \State $res.raiz \gets NULL$                                        \comment \bigo(1)
}{}

\algoritmo{iDefinir}
{   \param{in/out}{$e$}{diccTrie($\alpha$)},
    \param{in}{$c$}{string},
    \param{in}{$v$}{$\alpha$}}
{}{}
{
    \var $i : nat$
    \State $ i \gets 0$                                                 \comment \bigo(1)
    \var $p : puntero(nodo)$
    \var $n : nodo$
    \var $eraVacio : bool$

    \If{$e.raiz == NULL$}                                               \comment \bigo(1)
        \State $n \gets crearNodo()$                                    \comment \bigo(1)
        \State $e.raiz \gets \&n$                                       \comment \bigo(1)
        \State $eraVacio \gets true$                                    \comment \bigo(1)
    \Else
        \State $eraVacio \gets false$                                   \comment \bigo(1)
    \EndIf
    \State $ p \gets e.raiz$                                            \comment \bigo(1)

    \While{$i<(longitud(c))$}
        \Statex             \comment El loop se repite longitud de la clave veces
        \If{$p.hijos[ord(s[i])] == NULL$}                               \comment \bigo(1)
            \State $n \gets crearNodo()$                                \comment \bigo(1)
            \State $p.hijos[ord(s[i])] \gets \&n$                       \comment \bigo(1)
        \EndIf
        \State $p \gets p.hijos[ord(s[i])]$                             \comment \bigo(1)
        \State $i++$                                                    \comment \bigo(1)
    \EndWhile
        \State $*p.valor \gets v$                                       \comment \bigo(1)
    \If{$eraVacio \algOr c<e.minimo.clave$}                             \comment \bigo(1)
        \State $e.minimo.clave \gets c$                                 \comment \bigo(1)
        \State $e.minimo.valor \gets v$                                 \comment \bigo(1)
    \EndIf
    \If{$eraVacio \algOr c>e.maximo.clave$}                             \comment \bigo(1)
        \State $e.maximo.clave \gets c$                                 \comment \bigo(1)
        \State $e.maximo.valor \gets v$                                 \comment \bigo(1)
    \EndIf
    \State $AgregarAdelante(c, e.claves)$                               \comment \bigo(1)
    \State $p.clave \gets crearIt(e.claves)$                            \comment \bigo(1)

}    {  2* \bigo(1) + L * (5 * \bigo(1)) + 7 * \bigo(1) = \bigo(L) }

\algoritmo{iDefinido?}
{   \param{in/out}{$e$}{diccTrie($\alpha$)},
    \param{in}{$c$}{string}}
    {bool}
    {\bigo(L)}
{
    \var $i : nat$
    \var $p : puntero(nodo)$
    \State $ i \gets 0$                                                 \comment \bigo(1)

    \If{$e.raiz == NULL$}                                               \comment \bigo(1)
        \State $res \gets false$                                        \comment \bigo(1)
    \Else
        \State $ p \gets e.raiz$                                        \comment \bigo(1)
        \While{$i<(longitud(c)) \algAnd \neg res$}                      \comment \bigo(1)
            \Statex             \comment El loop se repite longitud de la clave veces o menos en caso de que res se vuelva false
             \If{$p.hijos[ord(s[i])] != NULL$}                          \comment \bigo(1)
                 \State $p \gets p.hijos[ord(s[i])]$                    \comment \bigo(1)
             \Else
                \State $res \gets false$                                \comment \bigo(1)
              \EndIf
             \State $i++$                                               \comment \bigo(1)
        \EndWhile
        \State $res \gets p.valor != NULL$                              \comment \bigo(1)
    \EndIf

}    {  2* \bigo(1) + L * (4 * \bigo(1)) + \bigo(1) = \bigo(L) }


\algoritmo{iObtener}
{   \param{in/out}{$e$}{diccTrie($\alpha$)},
    \param{in}{$c$}{string}}
    {$\alpha$}
    {\bigo(L)}
{
    \var $i : nat$
    \State $ i \gets 0$                                                 \comment \bigo(1)
    \var $p : puntero(nodo)$
    \State $ p \gets e.raiz$                                            \comment \bigo(1)
    \While{$i<(longitud(c))$}
         \Statex             \comment El loop se repite longitud de la clave veces
         \State $p \gets p.hijos[ord(s[i]))$                            \comment \bigo(1)
         \State $i++$                                                   \comment \bigo(1)
    \EndWhile
    \State $res \gets (*p).valor)$                                      \comment \bigo(1)

}    {  2* \bigo(1) + L * (2 * \bigo(1)) + \bigo(1) = \bigo(L) }

\algoritmo{iBorrar}
{   \param{in/out}{$e$}{diccTrie($\alpha$)},
    \param{in}{$c$}{string}}
    {}
    {\bigo(L)}
{
    \var $i : nat$
    \var $p : puntero(nodo)$
    \var $pi: pila(tupla(punte: puntero(nodo), siguiente: nat))$
    \State $ i \gets 0$                                                 \comment \bigo(1)
    \State $ p \gets e.raiz$                                            \comment \bigo(1)
    \While{$i<(longitud(c)-1)$}                                         \comment \bigo(1)
        \Statex             \comment El loop se repite longitud de la clave veces
        \State $p \gets p.hijos[ord(s[i]))$                             \comment \bigo(1)
        \State $apilar(pi, (p,s[i+1])) $                                \comment \bigo(copy(tupla(punte: puntero(nodo), siguiente: nat)))
        \State $i++$                                                    \comment \bigo(1)
    \EndWhile
    \State $EliminarSiguiente(p.clave)$                                 \comment \bigo(1)
    \State $p.clave \gets NULL$                                         \comment \bigo(1)
    \State $p \gets p.hijos[ord(s[i]))$                                 \comment \bigo(1)
    \State $apilar(pi, (p,NULL)) $                                      \comment \bigo(copy(tupla(punte: puntero(nodo), siguiente: nat)))
    \While{$(\neg TieneHijos(*(tope(pi).punte))) \algAnd ((*(tope(pi).punte)).clave==NULL)$}
                                                                        \comment \bigo(1)
        \Statex             \comment El loop se repite longitud de la clave veces como mucho

        \State $p \gets desapilar(pi)$                                  \comment \bigo(1)
        \State $(*(tope(pi).punte)).hijos[*(tope(pi).siguiente)] == NULL$   \comment \bigo(1)
    \EndWhile
    \If{$\neg tieneHijos(e.raiz)$}                                      \comment \bigo(1)
        \State $e.raiz \gets NULL$                                      \comment \bigo(1)
    \Else
        \State $ p \gets e.raiz$                                        \comment \bigo(1)
        \var $s : string$
        \State $ s \gets <>$                                            \comment \bigo(1)
        \While{$(*p).valor==NULL $}                                     \comment \bigo(1)
            \Statex             \comment El loop se repite L veces como mucho
            \State $s \gets s+chr(MenorHijo(*p))$                       \comment \bigo(1)
            \State $p \gets MenorHijo(*p)$                              \comment \bigo(1)

        \EndWhile
        \State $e.minimo.clave \gets s$                                 \comment \bigo(1)
        \State $e.minimo.valor \gets (*p).valor$                        \comment \bigo(1)
        \State $ s \gets <>$    \comment \bigo(1)
        \While{$TieneHijos(*p)$}                                        \comment \bigo(1)
            \Statex             \comment El loop se repite L veces como mucho
            \State $s \gets s+chr(MayorHijo(*p))$                       \comment \bigo(1)
            \State $p \gets MayorHijo(*p)$                              \comment \bigo(1)

        \EndWhile
        \State $e.maximo.clave \gets s$                                 \comment \bigo(1)
        \State $e.maximo.valor \gets (*p).valor$                        \comment \bigo(1)

    \EndIf

}    {  2* \bigo(1) + L * (2 * \bigo(1) + \bigo(copy(tupla(punte: puntero(nodo), siguiente: nat)))) + \bigo(1) +\bigo(copy(tupla(punte: puntero(nodo), siguiente: nat))))+ L * 2 * \bigo(1)+ 7 * \bigo(1)== \bigo(L) en caso de que el arbol este vacio, el el otro caso es
2* \bigo(1) + L * (2 * \bigo(1) + \bigo(copy(tupla(punte: puntero(nodo), siguiente: nat)))) + \bigo(1) +\bigo(copy(tupla(punte: puntero(nodo), siguiente: nat))))+ L * 2 * \bigo(1)+\bigo(1)+2* L * 2 * \bigo(1) +2 * 2 *big O(1) == \bigo(L)}

\algoritmo{iMaximo}{
\param{in/out}{$e$}{diccTrie($\alpha$)}
        }{tupla(clave: string, significado: $\alpha$)}{\bigo(1)}{
    \State $ res \gets e.maximo$                                         \comment \bigo(1)
}{}

\algoritmo{iMinimo}{
\param{in/out}{$e$}{diccTrie($\alpha$)}
        }{tupla(clave: string, significado: $\alpha$)}{\bigo(1)}{
    \State $ res \gets e.minimo$                                         \comment \bigo(1)
}{}





\subsubsection{Funciones auxiliares de los algoritmos}

\algoritmo{iCrearNodo}{
}{nodo}{\bigo(1)}
  {
    \var $d : arreglo_est\acute{a}tico[256] (puntero(nodo))$
    \var $i : nat$
    \State $i \gets 0)$                                                 \comment \bigo(1)

    \While{$i<256$}
         \Statex             \comment El loop se repite 256 veces
         \State $d[i] \gets NULL$                                       \comment \bigo(1)
         \State $i++$                                                   \comment \bigo(1)
    \EndWhile
    \State $res.hijos \gets d$                                          \comment \bigo(1)
    \State $res.valor \gets NULL$                                       \comment \bigo(1)
    \State $res.clave \gets NULL$                                       \comment \bigo(1)
}{   \bigo(1) + 256 * (2 * \bigo(1)) + 3 * \bigo(1) = \bigo(1) }

\algoritmo{iTieneHijos}{
\param{in}{$n$}{nodo}
}{bool}{\bigo(1)}
  {
    \var $i : nat$
    \State $i \gets 0$                                                  \comment \bigo(1)
    \State $res \gets false$                                            \comment \bigo(1)
    \While{$i<256$}
         \Statex             \comment El loop se repite 256 veces
         \If{$n.hijos[i] != NULL$}                                      \comment \bigo(1)
             \State $res \gets true$                                    \comment \bigo(1)
          \EndIf
    \EndWhile
}{  2* \bigo(1) + 256 * (2 * \bigo(1))  = \bigo(1) }

\algoritmo{iMenorHijo}{
\param{in}{$n$}{nodo}
}{nat}{\bigo(1)}
  {
    \var $i : nat$
    \State $i \gets 0$                                                  \comment \bigo(1)
    \State $res \gets 256$                                              \comment \bigo(1)
    \While{$i<256$}
         \Statex             \comment El loop se repite 256 veces
         \If{$n.hijos[i] != NULL \algAnd i<res$}                        \comment \bigo(1)
             \State $res \gets i$                                       \comment \bigo(1)
          \EndIf
    \EndWhile
}{  2* \bigo(1) + 256 * (2 * \bigo(1))  = \bigo(1) }

\algoritmo{iMayorHijo}{
\param{in}{$n$}{nodo}
}{nat}{\bigo(1)}
  {
    \var $i : nat$
    \State $i \gets 0$                                                  \comment \bigo(1)
    \State $res \gets 0$                                                \comment \bigo(1)
    \While{$i<256$}
         \Statex             \comment El loop se repite 256 veces
         \If{$n.hijos[i] != NULL \algAnd i>res$}                        \comment \bigo(1)
             \State $res \gets i$                                       \comment \bigo(1)
          \EndIf
    \EndWhile
}{  2* \bigo(1) + 256 * (2 * \bigo(1))  = \bigo(1) }


\subsection{Servicios usados}

\usaServicio{pila ($\alpha$)}
\usaServicio{arreglo$\_$estatico ($\alpha$)}
\usaServicio{AgregarAdelante(itLista($\alpha$))}
\usaServicio{EliminarSiguiente(itLista($\alpha$))}

