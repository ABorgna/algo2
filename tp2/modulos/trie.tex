\section{diccTrie($\alpha$)}

El módulo diccTrie provee un diccionario con claves de tipo String y acceso, inserción y borrado en \bigo($L$), donde $L$ es el largo máximo de las claves.

\subsection{Interfaz}

\begin{iparamformales}{$\alpha$}

\end{iparamformales}

\iusa{}
\iseexplica{Diccionario($string, \alpha$)}
\igenero{diccTrie($\alpha$)}
\ioperaciones

\operacion{NuevoDiccTrie}
{}
{diccTrie($\alpha$)}
{true}
{$res \igobs vacio$}
{\bigo(1)}
{}
{Crea un diccionario vacio}

\operacion{Definir}
{   \param{in/out}{$d$}{diccTrie($\alpha$)},
    \param{in}{$c$}{string},
    \param{in}{$v$}{$\alpha$}}
{}
{$d \igobs d_0$}
{$d \igobs definir(c, v, d_0)$}
{\bigo($L$)}
{}
{Modifica el diccionario agregando o reemplazando el significado de una clave 
    con un nuevo valor}

\operacion{Definido?}
{   \param{in}{$d$}{diccTrie($\alpha$)},
    \param{in}{$c$}{string}}
{bool}
{true}
{$res \igobs def?(c, d)$}
{\bigo($L$)}
{}
{Devuelve true si una clave se encuentra definida en el diccionario}

\operacion{Obtener}
{   \param{in}{$d$}{diccTrie($\alpha$)},
    \param{in}{$c$}{string}}
{$\alpha$}
{$def?(c, d)$}
{$res$ \igobs $obtener(c, d)$}
{\bigo($L$)}
{}
{Devuelve el significado definido para la clave $c$}

\operacion{Borrar}
{   \param{in/out}{$d$}{diccTrie($\alpha$)},
    \param{in}{$c$}{string}}
{}
{$d \igobs d_0 \land def?(c, d)$}
{$d \igobs borrar(c, d_0)$}
{\bigo($L$)}
{}
{Borra el significado asociado a la clave $c$}

\operacion{Maximo}
{   \param{in}{$d$}{diccLog($\alpha$)}}
{tupla(clave: string, significado: $\alpha$)}
{$\neg (d \igobs vacio)$}
{$siguiente(res) \igobs tupla(maximo(d), obtener(maximo(d)))$}
{\bigo(1)}
{}
{Obtiene una tupla de la clave y el significado del elemento con la clave 
    mas grande en el diccionario}

\operacion{Minimo}
{   \param{in}{$d$}{diccLog($\alpha$)}}
{tupla(clave: string, significado: $\alpha$)}
{$\neg (d \igobs vacio)$}
{$siguiente(res) \igobs tupla(minimo(d), obtener(minimo(d)))$}
{\bigo(1)}
{}
{Obtiene una tupla de la clave y el significado del elemento con la clave 
    mas pequeña en el diccionario}


\subsection{Representación}
\serepresenta{diccTrie($\alpha$)}{trie}
\donde{trie}{tupla(\\
    $raiz$: puntero(nodo),\\
    $minimo$: clavevalor,
    $maximo$: clavevalor }
\donde{nodo}{tupla(\\
    $valor$: $\alpha$,\\
    $hijos$: arreglo\_estático[256] (puntero(nodo)) )}
\donde{clavevalor}{tupla(\\
    $clave$: string,\\
    $valor$: $\alpha$ )}
    


\subsubsection{Invariante de representación}
\begin{Rep}{estrDic}{e}
  \repfunc{Raíz es distinto de NULL}
    {raiz != NULL}
  \repfunc{Existe un \'unico camino entre cada nodo y el nodo ra\'iz (no hay ciclos)}
    {noHayCiclos(e)}
  \repfunc{Todos los nodos hojas, es decir, todos los que tienen su arreglo hijos con todas sus posiciones en NULL, tienen que tener un valor distinto de NULL.}
    {todasLasHojasTienenValor(e)}
  \repfunc{En claves est\'a el camino que se recorre desde la ra\'z hasta cada nodo hoja}
    {hayHojas(e) \Rightarrow \vert e.claves \vert > 0 \\
        \land  (\forall c : string)(c \in caminosANodos(e)) \Rightarrow  ((\exists i: nat)( i \in \{0.. \vert e.claves \vert  \}) \Rightarrow (e.claves[i] = c))
    }
  \repfunc{Los nodos minimo y maximo son los correspondientes}
    {}

\end{Rep}

\subsubsection{Operaciones auxiliares del invariante de Representaci\'on}

  \tadOperacion{noHayCiclos}{puntero(nodo)}{bool}{}
  \tadAxioma{noHayCiclos($n,p$)}{($\exists$ n:nat)(($\forall$ c: string)($\vert$s$\vert$ = n $\Rightarrow$ leer($p,s$) = NULL))}
  \tadOperacion{leer}{puntero(nodo), string}{bool}{}
  \tadAxioma{leer($p,s$)}{\IF vacia?($s$) THEN p $\rightarrow$ valor ELSE {\IF p $\rightarrow$ hijos[prim(s)] = NULL THEN NULL ELSE leer(p $\rightarrow$ hijos[prim(s)], fin(s)) FI} FI}
  \tadOperacion{todosNull}{arreglo(puntero(nodo))}{bool}{}
  \tadAxioma{todosNull($a$)}{auxTodosNull($a, 0$)}
  \tadOperacion{auxTodosNull}{arreglo(puntero(nodo)), nat}{bool}{}
  \tadAxioma{auxTodosNull($a, i$)}{\IF i < $\vert$a$\vert$ THEN a[i] == NULL $\land$ auxTodosNull($a, i+1$) ELSE a[i].valor == NULL FI}
  \tadOperacion{esHoja}{puntero(nodo)}{bool}{}
  \tadAxioma{esHoja($p$)}{\IF p == NULL THEN false ELSE todosNull(p.hijos) FI}
  \tadOperacion{todasLasHojas}{puntero(nodo), nat}{conj(nodo)}{}
  \tadAxioma{todasLasHojas($p, n$)}{\IF p == NULL THEN false ELSE {\IF esHoja($p$) THEN Ag(*p, vacio) ELSE auxTodasLasHojas((*p).hijos, 256) FI} FI}

  \tadOperacion{auxTodasLasHojas}{arreglo(puntero(nodo)), nat}{conj(nodo)}{}
  \tadAxioma{auxTodasLasHojas($a, n$)}{hojasDeHijos($a,n,0$)}

  \tadOperacion{hojasDeHijos}{arreglo(puntero(nodo)), nat, nat}{conj(nodo)}{}
  \tadAxioma{hojasDeHijos($a, n, i$)}{\IF i = n THEN $\emptyset$ ELSE todasLasHojas(a[i]) $\cup$ hojasDeHijos($a, n, (i+1)$) FI}

  \tadOperacion{todasLasHojasTienenValor}{puntero(nodo)}{bool}{}
  \tadAxioma{todasLasHojasTienenValor($p$)}{auxTodasLasHojasTienenValor(todasLasHojas($p, 256$))}

  \tadOperacion{auxTodasLasHojasTienenValor}{arreglo(puntero(nodo))}{bool}{}
  \tadAxioma{auxTodasLasHojasTienenValor($a$)}{\IF |a| = 0 THEN true ELSE dameUno(a).valor != NULL $\land$ auxTodasLasHojasTienenValor(sinUno(a)) FI}

\tadOperacion{hayHojas}{puntero(nodo)}{bool}{}
Dada una estructura, indica si en la misma existe alg\'un nodo cuyo arreglo hijos tenga todas las posiciones NULL.
\tadOperacion{caminosANodos}{puntero(nodo)}{conj(string)}{}
Dada una estructura, devuelve un conjunto con el \'unico camino existente entre la raiz y cada hoja. El camino se obtiene de agregar la posici\'on del arreglo hijos por la cual hay que bajar en cada nivel de la estructura hasta llegar a la hoja, conviritendo en cada paso esa posici\'on en un \textsc{char} y junt\'andolos en un \textsc{string}.

\subsubsection{Función de abstracción}

\begin{ABS}{e}{estrDicc}{d}{dicc(string,$\alpha$)}

  \absfunc{}{(\forall c:string)(definido?(c,d)) = (\exists n: nodo)(n \in todasLasHojas(e)) n.valor != NULL}
  \absfunc{}{(\exists i:nat)(i \in \{0..\vert e.claves \vert  \})  \Rightarrow  e.claves[i] = c \yluego significado(c,d) = leer(e.clave).valor}
\end{ABS}

\subsection{Algoritmos}
\algoritmo{iNuevoDiccTrie}{
        }{diccTrie($\alpha$)}{\bigo(1)}{
    \var $n : nodo$    
    \State $ n \gets crearNodo()$                   		  			\comment \bigo(1)    
    \State $res.raiz \gets *n$                     						\comment \bigo(1)    
    \State $res.minimo.valor \gets NULL$                     			\comment \bigo(1)  
    \State $res.minimo.clave \gets NULL$                     			\comment \bigo(1) 
    \State $res.maximo.valor \gets NULL$                     			\comment \bigo(1)  
    \State $res.maximo.clave \gets NULL$                     			\comment \bigo(1)                             
}{6 * \bigo(1) = \bigo(1)}

\algoritmo{iDefinir}
{   \param{in/out}{$e$}{diccTrie($\alpha$)},
    \param{in}{$c$}{string},
    \param{in}{$v$}{$\alpha$}}
{}{}
{
	\var $i : nat$
	\State $ i \gets 0$  												\comment \bigo(1) 
	\var $p : puntero(nodo)$ 
	\State $ p \gets e.raiz$											\comment \bigo(1) 
	\var $n : nodo$ 
	\While{$i<(longitud(c))$}
   		\Statex             \comment El loop se repite longitud de la clave veces
     	 \If{$p.hijos[ord(s[i]] == NULL$}                     			\comment \bigo(1)
            \State $n \gets crearNodo()$      							\comment \bigo(1)
           	\State $p.hijos[ord(s[i]) \gets \&n$              			\comment \bigo(1)
     	 \EndIf
     	\State $p \gets p.hijos[ord(s[i]))$								\comment \bigo(1)
     	\State $i++$   													\comment \bigo(1) 
    \EndWhile		
	\State $*p.valor \gets v)$										\comment \bigo(1)
    \If{$c<trie.minimo.clave\vert\vert trie.maximo.clave==NULL $}                     					\comment \bigo(1)
           \State $trie.minimo.clave \gets c$     				 	\comment \bigo(1)
             \State $trie.minimo.valor \gets v$     				 \comment \bigo(1)
       \EndIf	
     \If{$c>trie.maximo.clave \vert\vert trie.maximo.clave==NULL $}                     					\comment \bigo(1)
           \State $trie.maximo.clave \gets c$     				 	\comment \bigo(1)
           \State $trie.maximo.valor \gets v$     				 	\comment \bigo(1)
      \EndIf														 

}	{  2* \bigo(1) + L * (5 * \bigo(1)) + 7 * \bigo(1) = \bigo(L) }

\algoritmo{iDefinido?}
{   \param{in/out}{$e$}{diccTrie($\alpha$)},
    \param{in}{$c$}{string}}
    {bool}
    {\bigo(L)}
{
	\var $i : nat$
	\State $ i \gets 0$  												\comment \bigo(1) 
	\var $p : puntero(nodo)$ 
	\State $ p \gets e.raiz$											\comment \bigo(1)  
	\State $ res \gets NULL$											\comment \bigo(1)  
	\While{$i<(longitud(c)) \&\& res==NULL$}
	\Statex             \comment El loop se repite longitud de la clave veces o menos en caso de que res se vuelva false
		 \If{$p.hijos[ord(s[i]] != NULL$}                     			\comment \bigo(1)
		 	\State $p \gets p.hijos[ord(s[i]))$							\comment \bigo(1)
		 \Else
            \State $res \gets false$  									\comment \bigo(1)
     	 \EndIf	
     	\State $i++$   													\comment \bigo(1) 
    \EndWhile		
	\If{$res == NULL$}                     							\comment \bigo(1)
		 \State $res \gets true$									\comment \bigo(1)	
	\EndIf												 

}	{  3* \bigo(1) + L * (4 * \bigo(1)) + 2 * \bigo(1) = \bigo(L) }


\algoritmo{iObtener}
{   \param{in/out}{$e$}{diccTrie($\alpha$)},
    \param{in}{$c$}{string}}
    {$\alpha$}
    {\bigo(L)}
{
	\var $i : nat$
	\State $ i \gets 0$  												\comment \bigo(1) 
	\var $p : puntero(nodo)$ 
	\State $ p \gets e.raiz$											\comment \bigo(1)  
	\While{$i<(longitud(c))$}
     	\Statex             \comment El loop se repite longitud de la clave veces
     	\State $p \gets p.hijos[ord(s[i]))$								\comment \bigo(1)
     	\State $i++$   													\comment \bigo(1) 
    \EndWhile		
	\State $res \gets (*p).valor)$										\comment \bigo(1)												 

}	{  2* \bigo(1) + L * (2 * \bigo(1)) + \bigo(1) = \bigo(L) }

\algoritmo{iBorrar}
{   \param{in/out}{$e$}{diccTrie($\alpha$)},
    \param{in}{$c$}{string}}
    {}
    {\bigo(L)}
{
	\var $i : nat$
	\State $ i \gets 0$  												\comment \bigo(1) 
	\var $p : puntero(nodo)$ 
	\State $ p \gets e.raiz$											\comment \bigo(1)  
	\var $pi: pila(tupla(punte: puntero(nodo), siguiente: nat))$ 
	\While{$i<(longitud(c)-1)$}
     	\Statex             \comment El loop se repite longitud de la clave veces
     	\State $p \gets p.hijos[ord(s[i]))$								\comment \bigo(1)
     	\State $apilar(pi, (p,s[i+1])) $								\comment \bigo(copy(tupla(punte: puntero(nodo), siguiente: nat))) 
     	\State $i++$   													\comment \bigo(1) 
    \EndWhile	
    \State $p \gets p.hijos[ord(s[i]))$								\comment \bigo(1)
     \State $apilar(pi, (p,NULL)) $							\comment \bigo(copy(tupla(punte: puntero(nodo), siguiente: nat)))	
   \While{$(\neq TieneHijos(*(tope(pi).punte))) \& \& ((*(tope(pi).punte)).clave==NULL)$}
     	\Statex             \comment El loop se repite longitud de la clave veces como mucho
 
     	\State $p \gets desapilar(pi)$								\comment \bigo(1)
     	  \State $(*(tope(pi).punte)).hijos[*(tope(pi).siguiente)] == NULL$							\comment \bigo(1)	
    \EndWhile	
    \If{$\neq tieneHijos(e.raiz)$}                     			\comment \bigo(1)
		 	\State $e.maximo \gets NULL$							\comment \bigo(1)
	\State $e.minimo \gets NULL$							\comment \bigo(1)
	\Else
         \State $ p \gets e.raiz$	\comment \bigo(1)
         \var $s : string$ 
         \While{$(*p).valor==NULL $}
     		\Statex             \comment El loop se repite L veces como mucho
     		\State $s \gets s+chr(MenorHijo(*p))$								\comment \bigo(1)	
     		\State $p \gets MenorHijo(*p)$								\comment \bigo(1)	
     				
    	\EndWhile		
    	\State $e.minimo.clave \gets s$								\comment \bigo(1)	
    	\State $e.minimo.valor \gets (*p).valor$								\comment \bigo(1)	
    	\While{$iTieneHijos(*p)$}
     		\Statex             \comment El loop se repite L veces como mucho
     		\State $s \gets s+chr(MayorHijo(*p))$								\comment \bigo(1)	
     		\State $p \gets MayorHijo(*p)$								\comment \bigo(1)	
     				
    	\EndWhile		
    	\State $e.maximo.clave \gets s$								\comment \bigo(1)	
    	\State $e.maximo.valor \gets (*p).valor$								\comment \bigo(1)	
    	
     \EndIf	 	
												 

}	{  2* \bigo(1) + L * (2 * \bigo(1) + \bigo(copy(tupla(punte: puntero(nodo), siguiente: nat)))) + \bigo(1) +\bigo(copy(tupla(punte: puntero(nodo), siguiente: nat))))+ L * 2 * \bigo(1)+ 5 * \bigo(1)== \bigo(L) en caso de que el arbol este cavio, el el otro caso es 
2* \bigo(1) + L * (2 * \bigo(1) + \bigo(copy(tupla(punte: puntero(nodo), siguiente: nat)))) + \bigo(1) +\bigo(copy(tupla(punte: puntero(nodo), siguiente: nat))))+ L * 2 * \bigo(1)+\bigo(1)+2* L * 2 * \bigo(1) +2 * 2 *big O(1) == \bigo(L)}

\algoritmo{iMaximo}{
\param{in/out}{$e$}{diccTrie($\alpha$)}
        }{tupla(clave: string, significado: $\alpha$)}{\bigo(1)}{  
    \State $ res \gets e.maximo$                   		  			\comment \bigo(1)                             
}{ \bigo(1) = \bigo(1)}

\algoritmo{iMinimo}{
\param{in/out}{$e$}{diccTrie($\alpha$)}
        }{tupla(clave: string, significado: $\alpha$)}{\bigo(1)}{  
    \State $ res \gets e.minimo$                   		  			\comment \bigo(1)                             
}{ \bigo(1) = \bigo(1)}





\subsubsection{Funciones auxiliares de los algoritmos}

\algoritmo{iCrearNodo}{
}{nodo}{\bigo(1)}
  {
    \var $d : arreglo_est\acute{a}tico[256] (puntero(nodo))$
    \var $i : nat$
    \State $i \gets 0)$                									\comment \bigo(1)

    \While{$i<256$}
     	\Statex             \comment El loop se repite 256 veces
     	\State $d[i] \gets NULL)$										\comment \bigo(1)
     	\State $i++$   													\comment \bigo(1) 
    \EndWhile						
    \State $res.hijos \gets d)$											\comment \bigo(1) 
    \State $res.valor \gets NULL)$   									\comment \bigo(1) 
}{   \bigo(1) + 256 * (2 * \bigo(1)) + 2 * \bigo(1) = \bigo(1) }

\algoritmo{iTieneHijos}{
\param{in}{$n$}{nodo}
}{bool}{\bigo(1)}
  {
    \var $i : nat$
    \State $i \gets 0)$                									\comment \bigo(1)
    \State $res \gets false$										\comment \bigo(1)
    \While{$i<256$}
     	\Statex             \comment El loop se repite 256 veces
 		\If{$n.hijos[i] != NULL$}                     			\comment \bigo(1)
		 	\State $res \gets true$										\comment \bigo(1)
     	 \EndIf	 													 
    \EndWhile						
}{  2* \bigo(1) + 256 * (2 * \bigo(1))  = \bigo(1) }

\algoritmo{iMenorHijo}{
\param{in}{$n$}{nodo}
}{bool}{\bigo(1)}
  {
    \var $i : nat$
    \State $i \gets 0)$                									\comment \bigo(1)
    \State $res \gets 256$												\comment \bigo(1)
    \While{$i<256$}
     	\Statex             \comment El loop se repite 256 veces
 		\If{$n.hijos[i] != NULL \&\& i<res$}                     			\comment \bigo(1)
		 	\State $res \gets i$										\comment \bigo(1)
     	 \EndIf	 													 
    \EndWhile						
}{  2* \bigo(1) + 256 * (2 * \bigo(1))  = \bigo(1) }

\algoritmo{iMayorHijo}{
\param{in}{$n$}{nodo}
}{bool}{\bigo(1)}
  {
    \var $i : nat$
    \State $i \gets 0)$                									\comment \bigo(1)
    \State $res \gets 0$												\comment \bigo(1)
    \While{$i<256$}
     	\Statex             \comment El loop se repite 256 veces
 		\If{$n.hijos[i] != NULL \&\& i>res$}                     			\comment \bigo(1)
		 	\State $res \gets i$										\comment \bigo(1)
     	 \EndIf	 													
    \EndWhile						
}{  2* \bigo(1) + 256 * (2 * \bigo(1))  = \bigo(1) }


\subsection{Servicios usados}

\usaServicio{pila ($\alpha$)}
\usaServicio{arreglo$\_$estatico ($\alpha$)}

