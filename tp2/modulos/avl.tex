\section{diccLog($\kappa, \alpha$)}

El módulo diccLog provee un diccionario con acceso, inserción y borrado en \bigo(log($n$)), donde $n$ es la cantidad de elementos actuales.

\subsection{Interfaz}

\begin{iparamformales}{$\kappa, \alpha$}

    \funcion{$\bullet = \bullet$} % Nombre
        {\param{in}{$k_0$}{$\kappa$}, \param{in}{$k_1$}{$\kappa$}} % Parametros
        {bool} % Tipo resultado
        {true} % Pre
        {res \igobs ($k_0 = k_1$)} % Post
        {$\Theta(equal(k_0, k_1))$} % Complejidad
        {} % Aliasing
        {Función de igualdad de $\kappa$'s} % Descripcion

    \funcion{$\bullet > \bullet$} % Nombre
        {\param{in}{$k_0$}{$\kappa$}, \param{in}{$k_1$}{$\kappa$}} % Parametros
        {bool} % Tipo resultado
        {true} % Pre
        {res \igobs ($k_0 > k_1$)} % Post
        {$\Theta(greater(k_0, k_1))$} % Complejidad
        {} % Aliasing
        {Función orden estricto de $\kappa$'s} % Descripcion

    \funcion{Copiar}
        {\param{in}{$k$}{$\kappa$}}
        {$\kappa$}
        {true}
        {$res \igobs k$}
        {\bigo($copy(k)$)}
        {}
        {Funcion de copia de $\kappa$'s}


\end{iparamformales}

\iusa{Lista Enlazada($\alpha$)}
\iseexplica{Diccionario($\kappa, \alpha$), Iterador Unidireccional Modificable( tupla($\kappa, \alpha$) )}
\igenero{diccLog($\kappa, \alpha$), itDiccLog($\kappa, \alpha$)}

\ioperaciones

\operacion{NuevoDiccLog}
{}
{diccLog($\kappa, \alpha$)}
{true}
{$res \igobs vacio$}
{\bigo(1)}
{}
{Crea un diccionario vacio}

\operacion{Definir}
{   \param{in/out}{$d$}{diccLog($\kappa, \alpha$)},
    \param{in}{$c$}{$\kappa$},
    \param{in}{$v$}{$\alpha$}}
{}
{$d \igobs d_0$}
{$d \igobs definir(c, v, d_0)$}
{\bigo($log(n)cmp + copy(c)$), donde $n = \#(claves(d))$ y $cmp = equal(c, c')+greater(c, c')$}
{alias($obtener(c, d) = v$), hasta que se redefina o se borre la clave}
{Modifica el diccionario agregando o reemplazando el significado de una clave 
    con un nuevo valor}

\operacion{Definido?}
{   \param{in}{$d$}{diccLog($\kappa, \alpha$)},
    \param{in}{$c$}{$\kappa$}}
{bool}
{true}
{$res \igobs def?(c, d)$}
{\bigo($log(n)cmp$), donde $n = \#(claves(d))$ y $cmp = equal(c, c')+greater(c, c')$}
{}
{Devuelve true si una clave se encuentra definida en el diccionario}

\operacion{Obtener}
{   \param{in}{$d$}{diccLog($\kappa, \alpha$)},
    \param{in}{$c$}{$\kappa$}}
{$\alpha$}
{$def?(c, d)$}
{$res$ \igobs $obtener(c, d)$}
{\bigo($log(n)cmp$), donde $n = \#(claves(d))$ y $cmp = equal(c, c')+greater(c, c')$}
{}
{Devuelve el significado definido para la clave $c$}

\operacion{Borrar}
{   \param{in/out}{$d$}{diccLog($\kappa, \alpha$)},
    \param{in}{$c$}{$\kappa$}}
{}
{$d \igobs d_0 \land def?(c, d)$}
{$d \igobs borrar(c, d_0)$}
{\bigo($log(n)cmp$), donde $n = \#(claves(d))$ y $cmp = equal(c, c')+greater(c, c')$}
{}
{Borra el significado asociado a la clave $c$}


\operacion{Maximo}
{   \param{in}{$d$}{diccLog($\kappa, \alpha$)}}
{tupla(clave: $\kappa$, significado: $\alpha$)}
{$\neg (d \igobs vacio)$}
{$siguiente(res) \igobs tupla(maximo(d), obtener(maximo(d)))$}
{\bigo(1)}
{}
{Obtiene una tupla de la clave y el significado del elemento con la clave 
    mas grande en el diccionario}

\operacion{Minimo}
{   \param{in}{$d$}{diccLog($\kappa, \alpha$)}}
{tupla(clave: $\kappa$, significado: $\alpha$)}
{$\neg (d \igobs vacio)$}
{$siguiente(res) \igobs tupla(minimo(d), obtener(minimo(d)))$}
{\bigo(1)}
{}
{Obtiene una tupla de la clave y el significado del elemento con la clave 
    mas pequeña en el diccionario}

\tadOperacion{minimo}
{diccLog($\kappa\ \alpha$)/d}
{$\kappa$}
{$\neg vacia?(claves(d))$}
\tadAxioma{minimo(d)}{$
    \IFLM len(claves(d)) = 1 THEN prim(claves(d)) ELSE  \\
    \hspace*{2em}    \IFLM prim(claves(d)) > minimo(borrar(prim(claves(d)), d)) THEN \\
    \hspace*{4em}        minimo(borrar(prim(claves(d)), d)) \\
    \hspace*{2em}    ELSE \\
    \hspace*{4em}        prim(claves(d)) \\
    \hspace*{2em}    FI \\
    FI
$}
\tadOperacion{maximo}
{diccLog($\kappa\ \alpha$)/d}
{$\kappa$}
{$\neg vacia?(claves(d))$}
\tadAxioma{maximo(d)}{$
    \IFLM len(claves(d)) = 1 THEN prim(claves(d)) ELSE  \\
    \hspace*{2em}    \IFLM prim(claves(d)) > maximo(borrar(prim(claves(d)), d)) THEN \\
    \hspace*{4em}        prim(claves(d)) \\
    \hspace*{2em}    ELSE \\
    \hspace*{4em}        maximo(borrar(prim(claves(d)), d)) \\
    \hspace*{2em}    FI \\
    FI
$}

\subsubsection{Operaciones del iterador}

\operacion{CrearIt}
{   \param{in}{$d$}{diccLog($\kappa, \alpha$)}}
{itDiccLog($\kappa, \alpha$)}
{true}
{$res \igobs crearItMod(<>, listaDeTupas(d))$ $\land$ alias($esPermutacion(SecuSuby(res), listaDeTuplas(d))$)}
{\bigo(1)}
{El iterador se invalida luego de cualquier operacion que modifique el arbol. No se puede
    asegurar con un dfs que el orden de los elementos en el arbol sea el mismo luego de un 
    balanceo.}
{}

\operacion{HayMas?}
{    \param{in}{$it$}{itDiccLog($\kappa, \alpha$)}}
{bool}
{true}
{$res \igobs hayMas?(it)$}
{\bigo(1)}
{}
{Devuelve true si hay elementos para avanzar}

\operacion{Actual}
{   \param{in}{$it$}{itDiccLog($\kappa, \alpha$)}}
{tupla(clave: $\kappa$, significado: $\alpha$)}
{$hayMas?(it)$}
{$res \igobs actual(it) \land$ alias($res.significado \igobs d.obtener(res.clave)$)}
{\bigo(1)}
{$res$.significado es una referencia al significado en el
    diccionario $d$ sobre el que se itera.}
{}

\operacion{Avanzar}
{   \param{in/out}{$it$}{itDiccLog($\kappa, \alpha$)}}
{}
{$it \igobs it_0 \land hayMas?(it)$}
{$it \igobs avanzar(it_0)$}
{\bigo(1)}
{}
{Avanza a la siguiente posición del iterador}

%\operacion{Eliminar}
%{   \param{in/out}{$it$}{itDiccLog($\kappa, \alpha$)}}
%{}
%{$it \igobs it_0 \land hayMas?(it)$}
%{$it \igobs eliminar(it_0)$}
%{\bigo($log(n)$), donde $n$ es la cantidad de elementos en el 
%    diccionario sobre el que se itera}
%{}
%{Elimina del diccionario la clave del siguiente elemento}


\tadOperacion{listaDeTuplas}
{diccLog($\kappa\ \alpha$)/d}
{secuencia(tupla($\kappa\ \alpha$))}
{}
\tadAxioma{listaDeTuplas(d)}{$
    \IFLM len(claves(d)) = 0 THEN <> ELSE  \\
    \hspace*{2em} <prim(claves(d)), obtener(prim(claves(d)))> \bullet\ listaDeTuplas(borrar(prim(claves(d)), d)) \\
    FI
$}


\subsection{Representación del diccionario}

\serepresenta{diccLog($\kappa, \alpha$)}{avl}
\donde{avl}{tupla(\\
    $raiz$: puntero(nodo),\\
    $max$: puntero(nodo), $min$: puntero(nodo) )}
\donde{nodo}{tupla(\\
    $clave$: $\kappa$, $valor$: $\alpha$,\\
    $menor$: nodo, $mayor$: nodo, $padre$: nodo\\
    $fdb$: nat )}


\subsubsection{Invariante de representación}

\begin{Rep}{$avl$}{$a$}
    \repfunc{ raiz es null (y max y min tambien son null) $\lor_L$ }
    {
        a.raiz = NULL \implies (a.max = NULL \land a.min = NULL)
    }
    \repfunc{ para cada nodo el factor de balanceo esta entre -1 y 1 }
    {
        fdbsEnRango(a.raiz)
    }
    \repfunc{ para cada nodo el factor de balanceo es igual a la diferencia 
        de altura de sus dos hijos }
    {
        fdbsCoinciden(a.raiz)
    }
    \repfunc{ para cada nodo, si tiene un nodo menor, su valor es mayor al 
        de todos los nodos (del maximo) de la rama menor }
    {
        menoresSonMenores(a.raiz)
    }
    \repfunc{ para cada nodo, si tiene un nodo mayor, su valor es menor al 
        de todos los nodos (del minimo) de la rama mayor }
    {
        mayoresSonMayores(a.raiz)
    }
    \repfunc{ el padre de raiz siembre debe ser null }
    {
        a.raiz \neq NULL \implies (*a.raiz).padre = NULL
    }
    \repfunc{ para cada nodo n, si tiene mayor o menor, sus padres debe ser n }
    {
        a.raiz \neq NULL \implies padresCorrectos(a.raiz)
    }
    \repfunc{ para cada dos nodos distintos, $n_1$ y $n_2$, sus hijos deben ser distintos }
    {
        verificarHijos(a.raiz, a.raiz)
    }
    \repfunc{ para cada nodo ninguno puede apuntar a la raiz}
    { }
    \repfunc{ max es igual al nodo con la clave mas grande }
    {
        a.raiz \neq NULL \implies a.max = \&maximo(a.raiz)
    }
    \repfunc{ min es igual al nodo con la clave mas pequeña }
    {
        a.raiz \neq NULL \implies a.min = \&minimo(a.raiz)
    }

\end{Rep}

\tadOperacion{fdbsEnRango}
{puntero(nodo)/n}
{bool}
{}
\tadAxioma{fdbsEnRango(n)}{$
    \IFLM n = NULL THEN true ELSE \\
    \hspace*{2em} -1 \leq (*n).fdb \leq 1 \land fdbsEnRango((*n).menor) \land fdbsEnRango((*n).mayor) \\
    FI
$}

\tadOperacion{fdbsCoinciden}
{puntero(nodo)/n}
{bool}
{}
\tadAxioma{fdbsCoinciden(n)}{$
    \IFLM n = NULL THEN true ELSE \\
    \hspace*{2em} (*n).fdb = altura((*n).menor) - altura((*n).mayor) \\
    \hspace*{2em} \land fdbsCoinciden((*n).menor) \land fdbsCoinciden((*n).mayor) \\
    FI
$}

\tadOperacion{atura}
{puntero(nodo)/n}
{nat}
{}
\tadAxioma{altura(n)}{$
    \IFLM n = NULL THEN 0 ELSE \\
    \hspace*{2em} 1 + max(altura((*n).menor), altura((*n).mayor)) \\
    FI
$}

\tadOperacion{menoresSonMenores}
{puntero(nodo)/n}
{bool}
{}
\tadAxioma{menoresSonMenores(n)}{$
    \IFLM n = NULL \lor (*n).menor = NULL THEN true ELSE \\
    \hspace*{2em} maximo((*n).menor).clave < (*n).clave \\
    \hspace*{2em} \land menoresSonMenores((*n).menor) \land menoresSonMenores((*n).mayor) \\
    FI
$}

\tadOperacion{minimo}
{puntero(nodo)/n}
{nodo}
{$n \neq NULL$}
\tadAxioma{minimo(n)}{$
    \IFLM (*n).menor = NULL THEN (*n) ELSE \\
    \hspace*{2em} minimo((*n).menor) \\
    FI
$}

\tadOperacion{mayoresSonMayores}
{puntero(nodo)/n}
{bool}
{}
\tadAxioma{mayoresSonMayores(n)}{$
    \IFLM n = NULL \lor (*n).mayor = NULL THEN true ELSE \\
    \hspace*{2em} minimo((*n).mayor).clave < (*n).clave \\
    \hspace*{2em} \land mayoresSonMayores((*n).menor) \land mayoresSonMayores((*n).mayor) \\
    FI
$}

\tadOperacion{maximo}
{puntero(nodo)/n}
{nodo}
{$n \neq NULL$}
\tadAxioma{maximo(n)}{$
    \IFLM (*n).mayor = NULL THEN (*n) ELSE \\
    \hspace*{2em} maximo((*n).mayor) \\
    FI
$}

\tadOperacion{padresCorrectos}
{puntero(nodo)/n}
{bool}
{$n \neq NULL$}
\tadAxioma{padresCorrectos(n)}{$
    ((*n).menor \neq NULL \implies ( (*n).menor.padre = n \land padresCorrectos((*n).menor) ) \\
    \land 
    ((*n).mayor \neq NULL \implies ( (*n).mayor.padre = n \land padresCorrectos((*n).mayor) )
$}

\tadOperacion{verificarHijos}
{puntero(nodo)/r, puntero(nodo)/n}
{bool}
{}
\tadAxioma{verificarHijos(r, n)}{$
    \IFLM r = NULL \lor n = NULL THEN true ELSE \\
    \hspace*{2em} (r \neq n \implies hijosDistintos(*r, *n)) \\
    \hspace*{2em} \land verificarHijos((*r).menor, n) \\
    \hspace*{2em} \land verificarHijos((*r).mayor, n) \\
    \hspace*{2em} \land verificarHijos(r, (*n).menor) \\
    \hspace*{2em} \land verificarHijos(r, (*n).mayor) \\
    FI
$}

\tadOperacion{hijosDistintos}
{nodo/a, nodo/b}
{bool}
{}
\tadAxioma{hijosDistintos(a, b)}{$
    a.menor \neq NULL \implies \\ 
    ( a.menor \neq a.mayor \land a.menor \neq b.menor \land a.menor \neq b.mayor ) \\
    a.mayor \neq NULL \implies \\
    ( a.mayor \neq b.menor \land a.mayor \neq b.mayor ) \\
    b.menor \neq NULL \implies \\
    ( b.menor \neq b.mayor )
$}




\subsubsection{Función de abstracción}

\begin{ABS}{$a$}{$avl$}{$d$}{$dicc$}
    \absfunc{}
    {
        (\forall c : \kappa)\ def?(c, d) \Leftrightarrow existe(c, a.raiz)\ \land_L
    }
    \absfunc{}
    {
        (\forall c : \kappa)\ def?(c, d) \implies obtener(c, d) = obtener(c, a.raiz)
    }
\end{ABS}

\tadOperacion{existe}
{$\kappa$/c, puntero(nodo)/n}
{bool}
{}
\tadAxioma{existe(c, n)}{$
    \IFLM n = NULL THEN false ELSE \\
    \hspace*{2em} (*n).clave = c \lor_L existe(c, (*n).menor) \lor existe(c, (*n).mayor) \\
    FI
$}

\tadOperacion{obtener}
{$\kappa$/c, puntero(nodo)/n}
{$\alpha$}
{existe(c, n)}
\tadAxioma{obtener(c, n)}{$
    \IFLM (*n).clave = c THEN (*n).valor ELSE \\
    \hspace*{2em} \IFLM existe(c, (*n).menor) THEN \\
    \hspace*{4em}     obtener(c, (*n).menor) \\
    \hspace*{2em} ELSE \\
    \hspace*{4em}     obtener(c, (*n).mayor) \\
    \hspace*{2em} FI \\
    FI
$}


\subsection{Representación del iterador}

\serepresenta{itDiccLog($\kappa, \alpha$)}{iter}
\donde{iter}{tupla(\\
    $actual$: puntero(nodo),\\
    $siguientes$: lista(puntero(nodo)),\\
    $dicc$: puntero(avl)}

\subsubsection{Invariante de representación del iterador}

\begin{Rep}{$iter$}{$it$}
    \repfunc{ El puntero al diccionario no puede ser null }
    {
        it.dicc \neq NULL \land_L
    }
    \repfunc{ Si el diccionario esta vacio no hay un elemento actual }
    {
        (*it.dicc).raiz = NULL \implies it.actual = NULL \land_L
    }
    \repfunc{ Si actual es null entonces no puede haber siguientes }
    {
        it.actual = NULL \implies it.siguientes = <> \land_L
    }
    \repfunc{ Si el diccionario no esta vacio y hay un elemento actual
        el elemento actual debe pertenecer al diccionario }
    {
        ((*it.dicc).raiz \neq NULL \land it.actual \neq NULL) \implies \\
        \hspace*{2em} pertenece(it.actual, (*it.dicc).raiz)
    }
    \repfunc{ Todos los elementos siguientes pertenecen al diccionario }
    {
        (\forall m : puntero(nodo), m \in it.siguientes) pertenece(m, (*it.dicc).raiz)
    }
    \repfunc{ Si hay un actual, sus hijos pertenecen a los elementos siguientes }
    {
        it.actual \neq NULL \implies \\
        \hspace*{2em} (*it.actual).menor \neq NULL \implies (*it.actual).menor \in it.siguientes \\
        \hspace*{2em} \land (*it.actual).menor \neq NULL \implies (*it.actual).menor \in it.siguientes
    }
\end{Rep}

\tadOperacion{pertenece}
{puntero(nodo)/a, puntero(nodo)/b}
{bool}
{$a \neq NULL$}

\tadAxioma{pertenece(a, b)}{$
    \IFLM b = NULL THEN false ELSE \\
    \hspace*{2em} b = a \lor_L pertenece(a, (*b).menor) \lor pertenece(a, (*b).mayor \\
    FI
$}

\subsubsection{Función de abstracción del iterador}

\begin{ABS}{$it$}{$iter$}{$im$}{$itMod$}
    \absfunc{}
    {
        anteriores(im) \igobs hasta(it.actual, dfs((*it.dicc).raiz))
    }
    \absfunc{}
    {
        siguientes(im) \igobs desde(it.actual, dfs((*it.dicc).raiz))
    }
\end{ABS}

\tadOperacion{hasta}
{$\gamma$/x, secuencia($\gamma$)/xs}
{secuencia($\gamma$)}
{}

\tadAxioma{hasta(x, xs)}{$
    \IFLM vacia?(xs) THEN <> ELSE \\
    \hspace*{2em} \IFLM prim(xs) = x THEN <> ELSE \\
    \hspace*{4em}     prim(xs) \bullet hasta(x, fin(xs)) \\
    \hspace*{2em} FI \\
    FI
$}

\tadOperacion{desde}
{$\gamma$/x, secuencia($\gamma$)/xs}
{secuencia($\gamma$)}
{}

\tadAxioma{desde(x, xs)}{$
    \IFLM vacia?(xs) THEN <> ELSE \\
    \hspace*{2em} \IFLM prim(xs) = x THEN xs ELSE \\
    \hspace*{4em}     desde(x, fin(xs)) \\
    \hspace*{2em} FI \\
    FI
$}

\tadOperacion{dfs}
{puntero(nodo)/n}
{secuencia(puntero(nodo))}
{}

\tadAxioma{dfs(n)}{$
    \IFLM n = NULL THEN <> ELSE \\
    \hspace*{2em} n \bullet dfs((*n).menor)\ \&\ dfs((*n).mayor) \\
    FI
$}


\subsection{Algoritmos}

\algoritmo{iNuevoDiccLog}
{}
{diccLog}
{\bigo(1)}
{   \State $res.raiz \gets NULL$                \comment \bigo(1)
    \State $res.min \gets NULL$                 \comment \bigo(1)
    \State $res.max \gets NULL$                 \comment \bigo(1)
}{3 * \bigo(1) = \bigo(1)}


\algoritmo{iMaximo}
{   \param{in}{$d$}{diccLog($\kappa, \alpha$)}
}
{tupla(clave: $\kappa$, significado: $\alpha$)}
{\bigo(1)}
{   \State $res \gets <clave: d.max.clave,\ significado: d.max.valor>$    \comment \bigo(1)
}{El acceso a las variables y la creacion de una tupla es \bigo(1)}


\algoritmo{iMinimo}
{   \param{in}{$d$}{diccLog($\kappa, \alpha$)}
}
{tupla(clave: $\kappa$, significado: $\alpha$)}
{\bigo(1)}
{   \State $res \gets <clave: d.min.clave,\ significado: d.min.valor>$    \comment \bigo(1)
}{El acceso a las variables y la creacion de una tupla es \bigo(1)}


\algoritmo{iDefinido?}
{   \param{in}{$d$}{diccLog($\kappa, \alpha$)},
    \param{in}{$c$}{$\kappa$}}
{bool}
{\bigo($log(N)(greater(\kappa, \kappa)+equal(\kappa, \kappa))$)}
{   \State $res \gets iauxDefinidoNodo(d.raiz, c)$          \comment \bigo($log(N)(greater(\kappa, \kappa)+equal(\kappa, \kappa))$)
}{Ver $iauxDefinidoNodo$.}

\algoritmo{iauxDefinidoNodo}
{   \param{in}{$n$}{puntero(nodo)},
    \param{in}{$c$}{$\kappa$}}
{bool}
{\bigo($log(N)(greater(\kappa, \kappa)+equal(\kappa, \kappa))$)}
{   \If{$n = NULL$}                                                     \comment \bigo(1)
        \State $res \gets false$                                        \comment \bigo(1)
    \Else                                                               \comment \bigo(1)
        \If{$c = (*n).clave$}                                           \comment \bigo($equal(\kappa, \kappa)$)
            \State $res \gets true$                                     \comment \bigo(1)
        \Else
            \If{$c > (*n).clave$}                                       \comment \bigo($greater(\kappa, \kappa)$)
                \State $res \gets auxDefinidoNodo((*n).mayor, c)$      \comment \bigo($(log(N)-1)(greater(\kappa, \kappa)+equal(\kappa, \kappa))$)
            \Else
                \State $res \gets auxDefinidoNodo((*n).menor, c)$      \comment \bigo($(log(N)-1)(greater(\kappa, \kappa)+equal(\kappa, \kappa))$)
            \EndIf
        \EndIf
    \EndIf
}{  Donde $N$ es la cantidad de elementos que hay insertados debajo de este nodo. 
    En cada llamada a $auxDefinidoNodo$ se reduce la cantidad de elementos aproximadamente a la mitad 
    haciendo una busqueda binaria y se compara la clave por igualdad y por orden. 
    Luego de log(N) llamadas se llega a las hojas del arbol terminando la busqueda. }


\algoritmo{iObtener}
{   \param{in}{$d$}{diccLog($\kappa, \alpha$)},
    \param{in}{$c$}{$\kappa$}}
{$\alpha$}
{\bigo($log(N)(greater(\kappa, \kappa)+equal(\kappa, \kappa))$)}
{   \State $res \gets (*auxObtenerNodo(d.raiz, c)).valor$               \comment \bigo($log(N)(greater(\kappa, \kappa)+equal(\kappa, \kappa))$)
}
{   Por complejidad ver $iauxObtenerNodo$. Como es condición necesaria para Obtener 
    que la clave este definida se puede asumir que $auxObtenerNodo$ devuelve un puntero no nulo. }

\algoritmo{iauxObtenerNodo}
{   \param{in}{$n$}{puntero(nodo))},
    \param{in}{$c$}{$\kappa$}}
{$\alpha$}
{\bigo($log(N)(greater(\kappa, \kappa)+equal(\kappa, \kappa))$)}
{   \If{$n = NULL$}                                                     \comment \bigo(1)
        \State $res \gets NULL$                                         \comment \bigo(1)
    \Else                                                               \comment \bigo(1)
        \If{$c = (*n).clave$}                                           \comment \bigo($equal(\kappa, \kappa)$)
            \State $res \gets n$                                        \comment \bigo(1)
        \Else
            \If{$c > (*n).clave$}                                       \comment \bigo($greater(\kappa, \kappa)$)
                \State $res \gets auxObtenerNodo((*n).mayor, c)$        \comment \bigo($(log(N)-1)(greater(\kappa, \kappa)+equal(\kappa, \kappa))$)
            \Else
                \State $res \gets auxObtenerNodo((*n).menor, c)$        \comment \bigo($(log(N)-1)(greater(\kappa, \kappa)+equal(\kappa, \kappa))$)
            \EndIf
        \EndIf
    \EndIf
}
{   Al igual que con $iauxDefinidoNodo$, $N$ es la cantidad de elementos debajo de este nodo.
    En cada llamada a $auxObtenerNodo$ se reduce la cantidad de elementos aproximadamente a la mitad y se
    compara la clave por igualdad y por orden. }


\algoritmo{iDefinir}
{   \param{in/out}{$d$}{diccLog($\kappa, \alpha$)},
    \param{in}{$c$}{$\kappa$},
    \param{in}{$v$}{$\alpha$}}
{}
{\bigo($copy(\kappa)+log(N)(greater(\kappa, \kappa)+equal(\kappa, \kappa))$)}
{   \If{$d.raiz = NULL$}                                                \comment \bigo(1)
        \State $d.raiz \gets auxNuevoNodo(c, v)$                        \comment \bigo($copy(\kappa)$)
    \Else
        \State $auxDefinirNodo(d.raiz, c, v)$                           \comment \bigo($copy(\kappa)+log(N)(greater(\kappa, \kappa)+equal(\kappa, \kappa))$) 
    \EndIf
}
{   Donde $N$ es la cantidad de elementos del arbol. La complejidad de
    definir una clave nueva es la de buscar si existe la misma ($log(N)$ comparaciones)
    y luego copiar la clave. El peor caso es en el que la clave no existe y hay que 
    bajar hasta las hojas del arbol para insertar un nuevo nodo.
    Incluso si el arbol esta vacio entonces $log(N) = 0$ por lo que sigue siendo
    $copy(\kappa)+log(N)(greater(\kappa, \kappa)+equal(\kappa, \kappa))$. }



\algoritmo{iauxNuevoNodo}
{   \param{in}{$c$}{$\kappa$},
    \param{in}{$v$}{$\alpha$},
    \param{in}{$p$}{puntero(nodo)}}
{puntero(nodo)}
{\bigo($copy(\kappa)$)}
{   \var $n : nodo$
    \State $n.clave \gets copy(c)$              \comment \bigo($copy(\kappa)$)
    \State $n.valor \gets v$                    \comment \bigo(1)
    \State $n.menor \gets NULL$                 \comment \bigo(1)
    \State $n.mayor \gets NULL$                 \comment \bigo(1)
    \State $n.padre \gets p$                    \comment \bigo(1)
    \State $n.fdb \gets 0$                      \comment \bigo(1)
    \State $res \gets \&n$                      \comment \bigo(1)
}
{   6*\bigo(1) + \bigo($copy(\kappa)$) = \bigo($copy(\kappa)$) }

\algoritmo{iauxDefinirNodo}
{   \param{in/out}{$d$}{diccLog($\kappa, \alpha$)},
    \param{in/out}{$n$}{puntero(nodo)},
    \param{in}{$c$}{$\kappa$},
    \param{in}{$v$}{$\alpha$}}
{}
{\bigo($copy(\kappa)+log(N)(greater(\kappa, \kappa)+equal(\kappa, \kappa))$)}
{   \If{$(*n).clave = c$}                                               \comment \bigo($equal(\kappa, \kappa)$))
        \State $(*n).valor \gets v$                                     \comment \bigo(1)
    \Else
        \If{$c > (*n).clave$}                                           \comment \bigo($greater(\kappa, \kappa)$)
            \If{$(*n).mayor = NULL$}                                    \comment \bigo(1)
                \State $(*n).mayor \gets auxNuevoNodo(c, v, n)$         \comment \bigo($copy(\kappa)$)
                \State $auxPropagarInsercion(d, (*n).mayor)$            \comment \bigo($log(N)$)
            \Else
                \State $auxDefinirNodo(d, (*n).mayor, c, v)$            \comment \bigo($copy(\kappa)+(log(N)-1)(greater(\kappa, \kappa)+equal(\kappa, \kappa))$)
            \EndIf
        \Else   
            \If{$(*n).menor = NULL$}                                    
                \State $(*n).menor \gets auxNuevoNodo(c, v, n)$         \comment \bigo($copy(\kappa)$)
                \State $auxPropagarInsercion(d, (*n).menor)$            \comment \bigo($log(N)$)
            \Else 
                \State $auxDefinirNodo(d, (*n).menor, c, v)$            \comment \bigo($copy(\kappa)+(log(N)-1)(greater(\kappa, \kappa)+equal(\kappa, \kappa))$)
            \EndIf
        \EndIf
    \EndIf
}
{   $log(N)$ veces \bigo($equal(\kappa, \kappa)$) $+$ \bigo($greater(\kappa, \kappa)$) $+$ \bigo(1)
    para realizar la busqueda de donde insertar la nueva clave y \bigo($copy(\kappa)$) $+$ \bigo($log(N)$)
    para crear el nodo y rebalancear el arbol. \\
    \hspace*{8em} \bigo($copy(\kappa)+log(N)$) $+$ \bigo($log(N)(greater(\kappa, \kappa)+equal(\kappa, \kappa))$) \\
    \hspace*{8em} \bigo($copy(\kappa)+log(N) + log(N)(greater(\kappa, \kappa)+equal(\kappa, \kappa))$) \\
    \hspace*{8em} \bigo($copy(\kappa)+log(N)(greater(\kappa, \kappa)+equal(\kappa, \kappa))$)
}


\algoritmo{iauxPropagarInsercion}
{   \param{in/out}{$d$}{diccLog($\kappa, \alpha$)},
    \param{in/out}{$n$}{puntero(nodo)}}
{}
{\bigo($log(N)$)}
{   \If{$ (*n).fdb > 1 \algOr (*n).fdb < -1 $}                          \comment \bigo(1)
        \State $auxBalancear(d, n)$                                     \comment \bigo(1)
    \Else
        \If{$ (*n).padre \neq NULL $}                                   \comment \bigo(1)
            \If{$ n = (*(*n).padre).menor $}                            \comment \bigo(1)
                \State $(*(*n).padre).fdb \gets (*(*n).padre).fdb + 1$  \comment \bigo(1)
            \Else
                \State $(*(*n).padre).fdb \gets (*(*n).padre).fdb - 1$  \comment \bigo(1)
            \EndIf

            \If{$ (*(*n).padre).fdb \neq 0 $}                           \comment \bigo(1)
                \State $auxPropagarBalance(d, (*n).padre)$              \comment \bigo($log(N)$)
            \EndIf
        \EndIf
    \EndIf
}
{   Por cada llamada recursiva sube un nivel en el arbol. Como la cantidad de niveles es
    $log(N)$, donde $N$ es la cantidad de elementos del arbol, en el peor de los casos
    hará $log(N)$ llamadas. \\
    \hspace*{8em} \bigo($log(N)$) $+$ \bigo(1)$*5 =$ \bigo($log(N)$)}


\algoritmo{iauxBalancear}
{   \param{in/out}{$d$}{diccLog($\kappa, \alpha$)},
    \param{in/out}{$n$}{puntero(nodo)}}
{}
{\bigo(1)}
{   \If{$ (*n).fdb < 0 $}                               \comment \bigo(1)
        \If{$ (*(*n).mayor).fdb > 0 $}                  \comment \bigo(1)
            \State $auxRotarDerecha(d, (*n).mayor)$     \comment \bigo(1)
        \EndIf
        \State $auxRotarIzquierda(d, n)$                \comment \bigo(1)
    \Else
        \If{$ (*(*n).menor).fdb > 0 $}                  \comment \bigo(1)
            \State $auxRotarIzquierda(a, (*n).menor)$   \comment \bigo(1)
        \EndIf
        \State $auxRotarDerecha(d, n)$                  \comment \bigo(1)
    \EndIf
}
{   Las rotaciones se realizan en \bigo(1) y a lo sumo debe realizar 2
    rotaciones para balancear el nodo. \\
    \hspace*{8em} \bigo(1) $=$ \bigo(1)$*4$}


\algoritmo{iauxRotarIzquierda}
{   \param{in/out}{$d$}{diccLog($\kappa, \alpha$)},
    \param{in/out}{$rr$}{puntero(nodo)}}
{}
{\bigo(1)}
{   \var $nr : puntero(nodo)$
    \State $nr \gets (*rr).mayor$                               \comment \bigo(1)
    \State $(*rr).mayor \gets (*nr).menor$                      \comment \bigo(1)
    \If{$ (*nr).menor \neq NULL $}                              \comment \bigo(1)
        \State $(*(*nr).menor).padre \gets rr$                  \comment \bigo(1)
    \EndIf
    \State $(*nr).padre \gets (*rr).padre$                      \comment \bigo(1)
    \If{$ (*rr).padre = NULL $}                                 \comment \bigo(1)
        \State $d.raiz \gets nr$                                \comment \bigo(1)
    \Else
        \If{$ (*(*rr).padre).menor = rr $}                      \comment \bigo(1)
            \State $(*(*rr).padre).menor \gets nr$              \comment \bigo(1)
        \Else
            \State $(*(*rr).padre).mayor \gets nr$              \comment \bigo(1)
        \EndIf
    \EndIf
    \State $(*nr).menor \gets rr$                               \comment \bigo(1)
    \State $(*rr).parent \gets nr$                              \comment \bigo(1)
    \State $(*rr).fdb \gets (*rr).fdb + 1 - min((*nr).fdb, 0)$  \comment \bigo(1)
    \State $(*nr).fdb \gets (*nr).fdb + 1 + min((*rr).fdb, 0)$  \comment \bigo(1)
}
{   Se realiza una cantidad de operaciones constante en el peor caso sobre punteros de nodos y
    naturales. \\
    \hspace*{8em} $14*$\bigo($1$) $=$ \bigo($1$). }

\algoritmo{iauxRotarDerecha}
{   \param{in/out}{$d$}{diccLog($\kappa, \alpha$)},
    \param{in/out}{$rr$}{puntero(nodo)}}
{}
{\bigo(1)}
{   \var $nr : puntero(nodo)$                                   
    \State $nr \gets (*rr).menor$                               \comment \bigo(1)
    \State $(*rr).menor \gets (*nr).mayor$                      \comment \bigo(1)
    \If{$ (*nr).mayor \neq NULL $}                              \comment \bigo(1)
        \State $(*(*nr).mayor).padre \gets rr$                  \comment \bigo(1)
    \EndIf
    \State $(*nr).padre \gets (*rr).padre$                      \comment \bigo(1)
    \If{$ (*rr).padre = NULL $}                                 \comment \bigo(1)
        \State $d.raiz \gets nr$                                \comment \bigo(1)
    \Else
        \If{$ (*(*rr).padre).menor = rr $}                      \comment \bigo(1)
            \State $(*(*rr).padre).menor \gets nr$              \comment \bigo(1)
        \Else
            \State $(*(*rr).padre).mayor \gets nr$              \comment \bigo(1)
        \EndIf
    \EndIf
    \State $(*nr).mayor \gets rr$                               \comment \bigo(1)
    \State $(*rr).parent \gets nr$                              \comment \bigo(1)
    \State $(*rr).fdb \gets (*rr).fdb - 1 + min((*nr).fdb, 0)$  \comment \bigo(1)
    \State $(*nr).fdb \gets (*nr).fdb - 1 - min((*rr).fdb, 0)$  \comment \bigo(1)
}
{   Se realiza una cantidad de operaciones constante sobre punteros de nodos y
    naturales. \\
    \hspace*{8em} $14*$\bigo($1$) $=$ \bigo($1$). }

\algoritmo{iBorrar}
{   \param{in/out}{$d$}{diccLog($\kappa, \alpha$)},
    \param{in}{$c$}{$\kappa$}}
{}
{\bigo($log(N)(greater(\kappa, \kappa)+equal(\kappa, \kappa))$)}
{   \State $auxBorrarNodo(d, d.raiz, c)$                        \comment \bigo($log(N)(greater(\kappa, \kappa)+equal(\kappa, \kappa))$)
}
{   Asumiendo que la clave $c$ esta definida puedo llamar a $auxBorrarNodo$. 
    La complejidad es la misma que la del auxiliar. 
}


\algoritmo{iauxBorrarNodo}
{   \param{in/out}{$d$}{diccLog($\kappa, \alpha$)},
    \param{in}{$n$}{puntero(nodo)},
    \param{in}{$c$}{$\kappa$}}
{}
{\bigo($log(N)(greater(\kappa, \kappa)+equal(\kappa, \kappa))$)}
{   \If{$ c = (*n).clave $}                                             \comment \bigo($equal(\kappa, \kappa)$)
        \If{$ (*n).menor \neq NULL \algAnd (*n).mayor \neq NULL $}      \comment \bigo(1)
            \var $n2 : puntero(nodo)$ 
            \State $n2 \gets auxMinimo((*n).mayor)$                     \comment \bigo(1)
            \State $(*n).clave \gets (*n2).clave$                       \comment \bigo(1)
            \State $(*n).valor \gets (*n2).valor$                       \comment \bigo(1)
            \State $auxRecortar(d, n2)$                                 \comment \bigo($log(N)$)
        \Else
            \State $auxRecortar(d, n)$                                  \comment \bigo($log(N)$)
        \EndIf
    \Else
        \If{$ c > (*n).clave $}                                         \comment \bigo($greater(\kappa, \kappa)$)
            \State $auxBorrarNodo(d, (*n).mayor, c)$                    \comment \bigo($(log(N)-1)(greater(\kappa, \kappa)+equal(\kappa, \kappa))$)
        \Else
            \State $auxBorrarNodo(d, (*n).menor, c)$                    \comment \bigo($(log(N)-1)(greater(\kappa, \kappa)+equal(\kappa, \kappa))$)
        \EndIf
    \EndIf
}
{   Debido a que se asume que la clave esta en el diccionario no hace falta 
    chequear si $n = NULL$. En cada llamado recursivo a la funcion se desciende
    un nivel en el arbol, realizando comparaciones sobre la clave $c$. Al encontrar el
    nodo, se encarga de borrarlo en \bigo($log(N)$). Como la cantidad de niveles del 
    arbol es $log(N)$, donde $N$ es la cantidad de elementos del mismo, realiza a lo 
    sumo $log(N)$ comparaciones. \\   
    \hspace*{8em} \bigo($log(N)$) $+$ \bigo($log(N)(greater(\kappa, \kappa)+equal(\kappa, \kappa))$) \\
    \hspace*{8em} \bigo($log(N) + log(N)(greater(\kappa, \kappa)+equal(\kappa, \kappa))$) \\
    \hspace*{8em} \bigo($log(N)(greater(\kappa, \kappa)+equal(\kappa, \kappa))$)
}

\algoritmo{iauxMinimo}
{   \param{in}{$n$}{puntero(nodo)}}
{puntero(nodo)}
{\bigo($log(N)$)}
{   \If{$ (*n).menor = NULL$}                           \comment \bigo(1)
        \State $res \gets n$                            \comment \bigo(1)
    \Else
        \State $res \gets auxMinimo((*n).menor)$        \comment \bigo($log(N)-1$)
    \EndIf
}
{   En cada llamada baja un nivel del arbol buscando el elemento minimo. Como la 
    cantidad de niveles del arbol es $log(N)$ donde $N$ es la cantidad de elementos, 
    en el peor caso debe descender por todo el arbol. \\
    \hspace*{8em} \bigo(1) + \bigo($log(N)$) $=$ \bigo($log(N)$) 
}
% \cuidado Aclarar que no puede recibir un puntero NULL


\algoritmo{iauxRecortar}
{   \param{in/out}{$d$}{diccLog($\kappa, \alpha$)},
    \param{in}{$n$}{puntero(nodo)}}
{}
{\bigo($log(N)$)}
{   \var $ho : puntero(nodo)$
    \If{$ (*n).menor \neq NULL $}                       \comment \bigo(1)
        \State $ho \gets (*n).menor$                    \comment \bigo(1)
        \State $(*ho).padre \gets (*n).padre$           \comment \bigo(1)
    \EndIf
    \If{$ (*n).mayor \neq NULL $}                       \comment \bigo(1)
        \State $ho \gets (*n).mayor$                    \comment \bigo(1)
        \State $(*ho).padre \gets (*n).padre$           \comment \bigo(1)
    \EndIf
    \If{$ (*n).padre = NULL $}                          \comment \bigo(1)
        \State $d.raiz \gets ho$                        \comment \bigo(1)
    \Else
        \State $auxPropagarBorrado(d, n)$               \comment \bigo($log(N)$)
        \If{$ (*(*n).padre).menor = n $}                \comment \bigo(1)
            \State $(*(*n).padre).menor \gets ho$       \comment \bigo(1)
        \Else
            \State $(*(*n).padre).mayor \gets ho$       \comment \bigo(1)
        \EndIf
    \EndIf
}
{   Como no puede ocurrir el caso de que el nodo tenga dos hijos, a lo sumo
    solo se entra en uno de los primeros ifs. \\
    \hspace*{8em.} \bigo(1)$*8 +$ \bigo($log(N)$) $=$ \bigo($log(N)$) }


\algoritmo{iauxPropagarBorrado}
{   \param{in/out}{$d$}{diccLog($\kappa, \alpha$)},
    \param{in}{$n$}{puntero(nodo)}}
{}
{\bigo($log(N)$)}
{   \If{$ (*n).fdb > 1 \algOr (*n).fdb < -1 $}                          \comment \bigo(1)
        \State $auxBalancear(n)$                                        \comment \bigo(1)
    \EndIf
    \If{$ (*n).padre \neq NULL $}                                       \comment \bigo(1)
        \If{$ (*(*n).padre).menor = n $}                                \comment \bigo(1)
            \State $(*(*n).padre).fdb \gets (*(*n).padre).fdb - 1$      \comment \bigo(1)
        \Else
            \State $(*(*n).padre).fdb \gets (*(*n).padre).fdb + 1$      \comment \bigo(1)
        \EndIf
        \State $auxPropagarBorrado(d, (*n).padre)$                      \comment \bigo($log(N)$)
    \EndIf
}
{   Por cada llamada recursiva sube un nivel en el arbol. Como la cantidad de niveles es
    $log(N)$, donde $N$ es la cantidad de elementos del arbol, en el peor de los casos
    hará $log(N)$ llamadas. \\
    \hspace*{8em} \bigo($log(N)$) $+$ \bigo(1)$*5 =$ \bigo($log(N)$) }


\subsubsection{Algoritmos del iterador}

\algoritmo{iCrearIt}
{   \param{in}{$d$}{diccLog($\kappa, \alpha$)}}
{}
{\bigo(1)}
{   \State $res.dicc \gets d$                                           \comment \bigo(1)
    \State $res.actual \gets d.raiz$                                    \comment \bigo(1)
    \State $res.siguientes \gets Vacia()$                               \comment \bigo(1)
    \If{$ *res.actual).menor \neq NULL $}                               \comment \bigo(1)
        \State $AgregarAdelante(res.siguientes, (*res.actual).menor)$   \comment \bigo(1)
    \EndIf
    \If{$ *res.actual).mayor \neq NULL $}                               \comment \bigo(1)
        \State $AgregarAdelante(res.siguientes, (*res.actual).mayor)$   \comment \bigo(1)
    \EndIf 
}
{   $7*$\bigo() $=$ \bigo(1) 
}

\algoritmo{iHayMas?}
{   \param{in}{$it$}{itDiccLog($\kappa, \alpha$)}}
{bool}
{\bigo(1)}
{   \State $res \gets it.actual \neq NULL$      \comment \bigo(1)
}
{}

\algoritmo{iActual}
{   \param{in}{$it$}{itDiccLog($\kappa, \alpha$)}}
{tupla($clave: \kappa, significado: \alpha$)}
{\bigo(1)}
{   \State $res \gets <clave: (*it.actual).clave,\ significado: (*it.actual).valor>$      \comment \bigo(1)
}
{}

\algoritmo{iAvanzar}
{   \param{in}{$it$}{itDiccLog($\kappa, \alpha$)}}
{}
{\bigo(1)}
{   \If{$ EsVacia?(res.siguientes) $}                                           \comment \bigo(1)
        \State $res.actual \gets NULL$                                          \comment \bigo(1)
    \Else
        \State $res.actual \gets Primero(res.siguientes)$                       \comment \bigo(1)
        \State $res.siguientes \gets Fin(res.siguientes)$                       \comment \bigo(1)
        \If{$ *res.actual).menor \neq NULL $}                                   \comment \bigo(1)
            \State $AgregarAdelante(res.siguientes, (*res.actual).menor)$       \comment \bigo(1)
        \EndIf
        \If{$ *res.actual).mayor \neq NULL $}                                   \comment \bigo(1)
            \State $AgregarAdelante(res.siguientes, (*res.actual).mayor)$       \comment \bigo(1)
        \EndIf 
    \EndIf
}
{   $7*$\bigo(1) $=$ \bigo(1)}


 
    
\subsection{Servicios usados}

Se utilizan las funciones exportadas por el modulo Lista Enlazada($\alpha$).
Se cuenta con que la complejidad de las operaciones Vacia, Primero, Fin y AgregarAdelante
sea \bigo(1).
