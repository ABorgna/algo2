\section{diccAVL($\kappa, \alpha$)}

El módulo diccAVL provee un diccionario con acceso, inserción y borrado en \bigo(log($n$)), donde $n$ es la cantidad de elementos actuales.

\subsection{Interfaz}

\begin{iparamformales}{$\kappa, \alpha$}

    \funcion{$\bullet = \bullet$} % Nombre
        {\param{in}{$k_0$}{$\kappa$}, \param{in}{$k_1$}{$\kappa$}} % Parametros
        {bool} % Tipo resultado
        {true} % Pre
        {res \igobs ($k_0 = k_1$)} % Post
        {$\Theta(equal(k_0, k_1))$} % Complejidad
        {} % Aliasing
        {Función de igualdad de $\kappa$'s} % Descripcion

    \funcion{$\bullet > \bullet$} % Nombre
        {\param{in}{$k_0$}{$\kappa$}, \param{in}{$k_1$}{$\kappa$}} % Parametros
        {bool} % Tipo resultado
        {true} % Pre
        {res \igobs ($k_0 > k_1$)} % Post
        {$\Theta(greater(k_0, k_1))$} % Complejidad
        {} % Aliasing
        {Función orden estricto de $\kappa$'s} % Descripcion

\end{iparamformales}

\iusa{}
\iseexplica{Diccionario($\kappa, \alpha$)}
\igenero{diccAvl($\kappa, \alpha$)}

\subsection{Representación}

\subsubsection{Invariante de representación}

\subsubsection{Función de abstracción}

\subsection{Algoritmos}

\subsection{Servicios usados}


