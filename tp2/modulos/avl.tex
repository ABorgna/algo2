\section{diccAVL($\kappa, \alpha$)}

El módulo diccAVL provee un diccionario con acceso, inserción y borrado en \bigo(log($n$)), donde $n$ es la cantidad de elementos actuales.

\subsection{Interfaz}

\begin{iparamformales}{$\kappa, \alpha$}

    \funcion{$\bullet = \bullet$} % Nombre
        {\param{in}{$k_0$}{$\kappa$}, \param{in}{$k_1$}{$\kappa$}} % Parametros
        {bool} % Tipo resultado
        {true} % Pre
        {res \igobs ($k_0 = k_1$)} % Post
        {$\Theta(equal(k_0, k_1))$} % Complejidad
        {} % Aliasing
        {Función de igualdad de $\kappa$'s} % Descripcion

    \funcion{$\bullet > \bullet$} % Nombre
        {\param{in}{$k_0$}{$\kappa$}, \param{in}{$k_1$}{$\kappa$}} % Parametros
        {bool} % Tipo resultado
        {true} % Pre
        {res \igobs ($k_0 > k_1$)} % Post
        {$\Theta(greater(k_0, k_1))$} % Complejidad
        {} % Aliasing
        {Función orden estricto de $\kappa$'s} % Descripcion

\end{iparamformales}

\iusa{}
\iseexplica{Diccionario($\kappa, \alpha$)}
\igenero{diccAVL($\kappa, \alpha$)}

\ioperaciones

\operacion{NuevoDiccAvl}
{}
{diccAVL($\kappa, \alpha$)}
{true}
{$res \igobs vacio$}
{\bigo(1)}
{}
{Crea un diccionario vacio}

\operacion{Definir}
{   \param{in/out}{$d$}{diccAVL($\kappa, \alpha$)},
    \param{in}{$c$}{$\kappa$},
    \param{in}{$v$}{$\alpha$}}
{}
{$d \igobs d_0$}
{$d \igobs definir(c, v, d_0)$}
{\bigo($log(n)$), donde $n = \#(claves(d))$}
{}
{Modifica el diccionario agregando o reemplazando el significado de una clave 
    con un nuevo valor}

\operacion{Def?}
{   \param{in}{$d$}{diccAVL($\kappa, \alpha$)},
    \param{in}{$c$}{$\kappa$}}
{bool}
{true}
{$res \igobs def?(c, d)$}
{\bigo($log(n)$), donde $n = \#(claves(d))$}
{}
{Devuelve true si una clave se encuentra definida en el diccionario}

\operacion{Obtener}
{   \param{in}{$d$}{diccAVL($\kappa, \alpha$)},
    \param{in}{$c$}{$\kappa$}}
{$\alpha$}
{$def?(c, d)$}
{$res$ \igobs $obtener(c, d)$}
{\bigo($log(n)$), donde $n = \#(claves(d))$}
{}
{Devuelve el significado definido para la clave $c$}



\subsection{Representación}

\subsubsection{Invariante de representación}

\subsubsection{Función de abstracción}

\subsection{Algoritmos}

\subsection{Servicios usados}


