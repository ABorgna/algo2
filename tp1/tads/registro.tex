\section{TAD \tadNombre{Registro}}

\begin{tad}{\tadNombre{Registro}}
\tadIgualdadObservacional{r}{r'}{registro}{rCampos(r) $\igobs$ rCampos(r') $\land$ (\paratodo{campo}{c})(c $\in$ rCampos(r) $\implies$ rValor(r, c) $\igobs$ rValor(r', c))}

\tadGeneros{registro}
\tadExporta{registro, generadores, observadores, otras operaciones}
\tadUsa{\tadNombre{Campo}, \tadNombre{Valor}, \tadNombre{Conj}}

\tadObservadores
%// Creo que si en vez de 'tieneCampo' guardamos la lista de campos
%\newline
%// nos es mas facil especificar los requiere de las inserciones (en db y en tabla)
%\newline
%// ya que seria igualdad de conjuntos directamente
\tadOperacion{rValor}{registro/r, campo/c}{valor}{rTieneCampo(r,c)}
%%\tadOperacion{rTieneCampo}{registro, campo}{bool}{} Lo cambiamos por rCampos;
\tadOperacion{rCampos}{registro/r}{conj(campo)}{}

\tadGeneradores
\tadOperacion{nuevoRegistro}{}{registro}{}
\tadOperacion{agregarValor}{registro/r, campo/c, valor/v}{registro}
    {$\neg$ rTieneCampo(r,c) $\land$ cEsString?(c) == vEsString?(v)}

\tadAxiomas[\paratodo{registro}{r}, \paratodo{valor}{v}, \paratodo{campo}{c}]
\tadAxioma{rValor(agregarValor(r, c, v), c')}{%
	\IF c == c' THEN%
		v%
	ELSE%
		rValor(r, c')
	FI%
}
\tadAxioma{rCampos(nuevoRegistro)}{$\varnothing$}
\tadAxioma{rCampos(agregarValor(r, c, v))}{Ag(c, campos(r))}

\end{tad}

